\documentclass[12pt, a4paper]{ctexart}
\usepackage[margin=2cm]{geometry}
\usepackage{libertine}

\usepackage{titlesec}
\usepackage{zhnumber}
\titleformat*{\section}{\Large\bfseries\raggedright}
\renewcommand{\thesection}{\zhnum{section}}
\renewcommand{\thesubsection}{\arabic{subsection}}

\setcounter{secnumdepth}{4}
\renewcommand{\theparagraph}{\thesubsubsection.\arabic{paragraph}}
\renewcommand{\paragraph}[1]{
    \refstepcounter{paragraph}
    {\bfseries\theparagraph\quad#1}\par
    \vspace{2pt}
    \noindent
}

\usepackage{enumitem}
\setlist[enumerate]{itemsep=2pt, parsep=0pt, partopsep=0pt, topsep=2pt}
\setlist[itemize]{itemsep=2pt, parsep=0pt, partopsep=0pt, topsep=2pt}
\linespread{1.2}

\usepackage{graphicx}

\usepackage{minted}
\setminted{
    breaklines=true, escapeinside=||, fontsize=\small, frame=lines,
    mathescape=$$, numbers=none, style=bw, tabsize=4
}

\usepackage{siunitx}

\begin{document}
    
\pagestyle{plain}
\thispagestyle{empty}

\noindent
\begin{tabular*}{\textwidth}{l @{\extracolsep{\fill}} r @{\extracolsep{6pt}} l}
    \LARGE{\textbf{实验报告}} & 计算机网络 & \textit{Computer Networking} \\
\end{tabular*}\\\\
\begin{tabular*}{\textwidth}{l l}
    \textbf{报告标题: } & \textbf{路由器及其动态路由协议} \\
    \textbf{学号: } & 19240212 \\
    \textbf{姓名: } & 华博文 \\
    \textbf{日期: } & 2025 年 11 月 8 日 \\
\end{tabular*}\\
\rule[2ex]{\textwidth}{2pt}

\section{实验目的}

本实验旨在掌握路由器动态路由(RIP)的配置方法,理解静态路由与动态 RIP 路由在配置方式、更新机制及适用场景上的区别;深入理解 RIP 协议的工作原理,包括路由信息的周期性交换、超时收敛机制;通过完整的网络拓扑搭建、设备配置及连通性测试流程,实现多网段主机间的通信,强化对动态路由协议在复杂网络中自动化管理作用的认知。

\section{实验内容简要描述}

\begin{enumerate}
    \item \textbf{路由器动态路由基础配置}:搭建包含 2 台交换机、2 台路由器、2 台主机的基础网络拓扑,完成交换机 VLAN 划分、路由器接口 IP 配置及静态路由设置,验证跨网段连通性;
    
    \item \textbf{动态路由(RIP)配置}:扩展网络拓扑至多 VLAN、多网段结构,在路由器上配置子接口实现 VLAN 间路由,启用 RIP 协议并完成网络通告,验证动态路由表生成及全网连通性;
    
    \item \textbf{静态路由与动态 RIP 路由对比分析}:从配置流程、拓扑适应性、路由更新方式三个维度,对比两种路由技术的核心差异;
    
    \item \textbf{RIP 工作原理验证}:通过观察路由表动态变化、模拟接口故障后的收敛过程,验证 RIP 协议的信息交换与超时机制。
\end{enumerate}

\section{实验步骤与结果分析}

\subsection{路由器动态路由基础配置}

\subsubsection{搭建基础网络拓扑}

在 Packet Tracer 中拖拽 2 台交换机(Switch0、Switch1)、2 台路由器(Router0、Router1)、4 台主机(PC0、PC1、PC2、PC3),按如下逻辑连接设备:

\begin{enumerate}
    \item Switch0 的以太网端口连接 PC0、PC1 和 Router0 的 FastEthernet0/0 接口;

    \item Switch1 的以太网端口连接 PC2、PC3 和 Router1 的 FastEthernet0/0 接口;

    \item Router0 与 Router1 通过 Serial0/1/0 串行接口互联;
\end{enumerate}

完成拓扑搭建后,呈现设备间的物理连接关系。

\begin{figure}[H]
    \centering
    \includegraphics[width=\textwidth]{./img/screenshot_0000.png}
\end{figure}

\subsubsection{配置交换机 VLAN 与接入端口}

在 Switch0 进入全局配置模式,创建 VLAN10,将 FastEthernet0/1 端口配置为接入模式并归属 VLAN10,同时配置 VLAN10 的 SVI 接口 IP 为 \mintinline{text}`192.168.100.100/24`,启用接口使其处于工作状态;Switch1 的配置同理,创建 VLAN20,将 FastEthernet0/1 端口配置为接入模式并归属 VLAN20,配置 VLAN20 的 SVI 接口 IP 为 \mintinline{text}`192.168.200.100/24`,启用接口。

\begin{figure}[H]
    \centering
    \includegraphics[width=\textwidth]{./img/screenshot_0001.png}
\end{figure}

\begin{figure}[H]
    \centering
    \includegraphics[width=\textwidth]{./img/screenshot_0002.png}
\end{figure}

通过此操作,实现交换机对连接主机的 VLAN 划分,为后续网段隔离奠定基础。

\subsubsection{配置路由器接口与静态路由}

在 Router0 配置 FastEthernet0/0 接口 IP 为 \mintinline{text}`192.168.100.1/24`,Serial0/1/0 接口 IP 为 \mintinline{text}`10.0.0.1/8`,并启用接口;Router1 配置 FastEthernet0/0 接口 IP 为 \mintinline{text}`192.168.200.1/24`,Serial0/1/0 接口 IP 为 \mintinline{text}`10.0.0.2/8`,并启用接口。

\begin{figure}[H]
    \centering
    \includegraphics[width=\textwidth]{./img/screenshot_0003.png}
\end{figure}

\begin{figure}[H]
    \centering
    \includegraphics[width=\textwidth]{./img/screenshot_0004.png}
\end{figure}

此步骤完成路由器的接口寻址与跨网段静态路由指引。

\subsubsection{配置主机 IP 与网关}

在 PC0 的网络配置界面,设置 IP 地址为 \mintinline{text}`192.168.100.10/24`,默认网关为 \mintinline{text}`192.168.100.1`;在 PC1 的网络配置界面,设置 IP 地址为 \mintinline{text}`192.168.100.12/24`;在 PC2 的网络配置界面,设置 IP 地址为 \mintinline{text}`192.168.200.11/24`,默认网关为 \mintinline{text}`192.168.200.1`。

\begin{figure}[H]
    \centering
    \includegraphics[width=\textwidth]{./img/screenshot_0005.png}
\end{figure}

\begin{figure}[H]
    \centering
    \includegraphics[width=\textwidth]{./img/screenshot_0006.png}
\end{figure}

确保主机的 IP 配置与所在 VLAN 的网段及网关逻辑匹配。

\subsubsection{基础网络连通性测试}

在 PC0 的命令行界面:

\begin{itemize}
    \item 执行 \mintinline{text}`ping 192.168.100.12` 命令,验证数据包能否通过 Switch0 转发,最终到达 PC1;
    
    \item 执行 \mintinline{text}`ping 192.168.100.100` 命令,验证数据包能否到达 Switch0;
\end{itemize}

\begin{figure}[H]
    \centering
    \includegraphics[width=\textwidth]{./img/screenshot_0007.png}
\end{figure}

\begin{itemize}
    \item 执行 \mintinline{text}`ping 192.168.100.1` 命令,验证数据包能否到达 PC0 网关;
    
    \item 执行 \mintinline{text}`ping 192.168.200.1` 命令,验证数据包能否到达对方路由器;
\end{itemize}

\begin{figure}[H]
    \centering
    \includegraphics[width=\textwidth]{./img/screenshot_0008.png}
\end{figure}

\begin{itemize}
    \item 执行 \mintinline{text}`ping 192.168.200.11` 命令,验证数据包能否通过路由在 Router0 与 Router1 间转发,最终到达 PC2。
\end{itemize}

\begin{figure}[H]
    \centering
    \includegraphics[width=\textwidth]{./img/screenshot_0009.png}
\end{figure}

若返回连续的 “Reply” 信息,说明基础网络连通性正常。

\subsection{动态路由(RIP)配置与验证}

\subsubsection{扩展多 VLAN 网络拓扑}

调整 Switch0 和 Switch1 的端口连接:

\begin{itemize}
    \item Switch0 的 FastEthernet0/2、0/3 端口分别连接 PC0、PC1,对应划分 VLAN2(\mintinline{text}`192.168.2.0/24`)、VLAN3(\mintinline{text}`192.168.3.0/24`);
    
    \item Switch1 的 FastEthernet0/2、0/3 端口分别连接 PC2、PC3,对应划分 VLAN4(\mintinline{text}`192.168.4.0/24`)、VLAN5(\mintinline{text}`192.168.5.0/24`);
    
    \item Switch0 与 Switch1 的 FastEthernet0/1 端口分别以 trunk 模式连接 Router0、Router1 的 FastEthernet0/0 接口;完成多 VLAN、多网段的拓扑扩展。
\end{itemize}

\subsubsection{配置交换机动态 VLAN 与 Trunk 链路}

配置 Switch0,创建 VLAN2、VLAN3,将 FastEthernet0/2 端口归属 VLAN2、FastEthernet0/3 端口归属 VLAN3;配置 VLAN2 的 SVI 接口 IP 为 \mintinline{text}`192.168.2.254/24`、VLAN3 的 SVI 接口 IP 为 \mintinline{text}`192.168.3.254/24` 并启用;将 FastEthernet0/1 端口配置为 trunk 模式,允许所有 VLAN 通过。

同样方法配置 Switch1,创建 VLAN4、VLAN5,将 FastEthernet0/2 端口归属 VLAN4、FastEthernet0/3 端口归属 VLAN5;配置 VLAN4 的 SVI 接口 IP 为 \mintinline{text}`192.168.4.254/24`、VLAN5 的 SVI 接口 IP 为 \mintinline{text}`192.168.5.254/24` 并启用;将 FastEthernet0/1 端口配置为 trunk 模式,允许所有 VLAN 通过。

\begin{figure}[H]
    \centering
    \includegraphics[width=\textwidth]{./img/screenshot_0010.png}
\end{figure}

\begin{figure}[H]
    \centering
    \includegraphics[width=\textwidth]{./img/screenshot_0011.png}
\end{figure}

此步骤实现交换机对多 VLAN 的支持及 trunk 链路的配置。

\subsubsection{配置路由器子接口与 RIP 协议}

进入 Router0 配置 FastEthernet0/0 接口的子接口模式,分别创建 Fa0/0.1(封装 VLAN2,IP 为 \mintinline{text}`192.168.2.1/24`)、Fa0/0.2(封装 VLAN3,IP 为 \mintinline{text}`192.168.3.1/24`)并启用;配置 Serial0/1/0 接口 IP 为 \mintinline{text}`10.0.0.1/24` 并启用;进入 RIP 协议配置模式,通告 \mintinline{text}`192.168.2.0`、\mintinline{text}`192.168.3.0`、\mintinline{text}`10.0.0.0`网络。

\begin{figure}[H]
    \centering
    \includegraphics[width=\textwidth]{./img/screenshot_0012.png}
\end{figure}

同理,进入 Router1 创建 Fa0/0.1(封装 VLAN4,IP 为 \mintinline{text}`192.168.4.1/24`)、Fa0/0.2(封装 VLAN5,IP 为 \mintinline{text}`192.168.5.1/24`)并启用;配置 Serial0/1/0 接口 IP 为 \mintinline{text}`10.0.0.2/24` 并启用;进入 RIP 协议配置模式,通告 \mintinline{text}`192.168.4.0`、\mintinline{text}`192.168.5.0`、\mintinline{text}`10.0.0.0` 网络。

\begin{figure}[H]
    \centering
    \includegraphics[width=\textwidth]{./img/screenshot_0013.png}
\end{figure}

通过子接口与 RIP 配置,实现路由器对多 VLAN 的路由支持及动态路由信息交换。

\subsubsection{动态路由网络连通性测试}

在 PC0 的命令行界面:

\begin{itemize}
    \item 执行 \mintinline{text}`ping 192.168.2.100` 命令,验证数据包能否到达 Switch0;
    
    \item 执行 \mintinline{text}`ping 192.168.5.1` 命令,验证数据包能否到达 Switch1;
    
    \item 执行 \mintinline{text}`ping 192.168.4.11` 命令,验证数据包能否通过 RIP 动态路由在多网段、多 VLAN 间转发,最终到达 PC2。
\end{itemize}

若返回连续 “Reply” 信息,说明动态路由配置生效,全网连通性正常。

\begin{figure}[H]
    \centering
    \includegraphics[width=\textwidth]{./img/screenshot_0014.png}
\end{figure}

\begin{figure}[H]
    \centering
    \includegraphics[width=\textwidth]{./img/screenshot_0015.png}
\end{figure}

\section{实验中遇到的问题及体会}

在配置路由器子接口与交换机 trunk 链路时,因遗漏子接口的 \mintinline{text}`encapsulation dot1q` VLAN 封装命令,导致 Router0 与 Switch0 的 VLAN2 数据链路无法建立。排查时通过对比接口配置、测试单网段连通性,最终补充封装命令后解决问题。这一过程让我深刻认识到:网络分层架构中,数据链路层的 VLAN 封装与网络层的 IP 配置需严格协同,任何一层的配置缺失或不匹配都会导致通信中断。

通过本次实验,我对动态路由协议(RIP)的自动化价值有了直观认知。静态路由虽配置逻辑简单,但在多网段场景下维护成本极高;而 RIP 协议通过自动交换路由信息,大幅降低了管理员的配置工作量,尤其在拓扑变化时的自收敛特性,体现了动态路由在复杂网络中的核心优势。

此外,实验深化了我对 “网络协议分层协作” 的理解。从交换机的 VLAN 划分(数据链路层)、路由器的子接口路由(网络层),到 RIP 的路由信息交换(应用层协议),每一层技术都在各自维度解决网络通信问题,且层与层之间通过协议封装形成协作关系。这种 “分层解耦、各司其职” 的设计思想,不再是理论概念,而是通过实际配置与测试可感知的工程逻辑。

最后,Packet Tracer 的虚拟环境为实验提供了安全的 “试错” 空间。在真实网络中配置错误可能引发业务中断,而在虚拟环境中可反复调整配置、模拟故障,这一过程不仅提升了我的命令行操作熟练度,更培养了从现象倒推配置逻辑的故障排查思维,为后续网络技术的学习与实践奠定了坚实基础。

\end{document}