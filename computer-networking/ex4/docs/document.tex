\documentclass[12pt, a4paper]{ctexart}
\usepackage[margin=2cm]{geometry}
\usepackage{libertine}

\usepackage{titlesec}
\usepackage{zhnumber}
\titleformat*{\section}{\Large\bfseries\raggedright}
\renewcommand{\thesection}{\zhnum{section}}
\renewcommand{\thesubsection}{\arabic{subsection}}

\setcounter{secnumdepth}{4}
\renewcommand{\theparagraph}{\thesubsubsection.\arabic{paragraph}}
\renewcommand{\paragraph}[1]{
    \refstepcounter{paragraph}
    {\bfseries\theparagraph\quad#1}\par
    \vspace{2pt}
    \noindent
}

\usepackage{enumitem}
\setlist[enumerate]{itemsep=2pt, parsep=0pt, partopsep=0pt, topsep=2pt}
\setlist[itemize]{itemsep=2pt, parsep=0pt, partopsep=0pt, topsep=2pt}
\linespread{1.2}

\usepackage{graphicx}

\usepackage{minted}
\setminted{
    breaklines=true, escapeinside=||, fontsize=\small, frame=lines,
    mathescape=$$, numbers=none, style=bw, tabsize=4
}

\usepackage{siunitx}

\begin{document}
    
\pagestyle{plain}
\thispagestyle{empty}

\noindent
\begin{tabular*}{\textwidth}{l @{\extracolsep{\fill}} r @{\extracolsep{6pt}} l}
    \LARGE{\textbf{实验报告}} & 计算机网络 & \textit{Computer Networking} \\
\end{tabular*}\\\\
\begin{tabular*}{\textwidth}{l l}
    \textbf{报告标题: } & \textbf{协议分析——ARP 协议} \\
    \textbf{学号: } & 19240212 \\
    \textbf{姓名: } & 华博文 \\
    \textbf{日期: } & 2025 年 9 月 24 日 \\
\end{tabular*}\\
\rule[2ex]{\textwidth}{2pt}

\section{实验目的}

本实验旨在掌握 Wireshark 的启动、网卡选择、数据包捕获与筛选操作, 熟悉其主窗口三部分结构及功能, 同时理解 HTTP 协议的工作流程并能分析 HTTP 数据包在数据链路层、网络层、传输层及应用层的关键字段信息;掌握 ARP 协议实现 IP 地址到 MAC 地址映射的核心功能、ARP 报文的结构 (字段组成、分层特点) 及 ``请求——响应'' 机制, 学会使用 C++ 语言按照 ARP 协议格式解析指定十六进制数据包并提取规范输出 ARP 报文关键字段, 最终加深对计算机网络分层模型 (数据链路层、网络层、传输层、应用层) 的理解, 验证 ``协议封装与解封装'' 的实际过程.

\section{实验内容简要描述}

\begin{enumerate}
    \item Wireshark 与 HTTP 协议分析: 通过终端启动 Wireshark, 选择目标网卡 (如 \mintinline{text}|eth0|), 启动抓包后用 Chrome 浏览器访问南京师范大学官网, 页面加载完成后停止抓包, 使用 \mintinline{text}|http| 筛选器筛选 HTTP 报文, 分析报文的分层协议信息 (源 / 目的 MAC、源 / 目的 IP、TCP 端口、HTTP 请求 / 响应内容等);
    
    \item ARP 协议分析: 在 Wireshark 中使用 \mintinline{text}|arp| 筛选器显示 ARP 报文, 分析 ARP 报文的分层结构 (仅数据链路层与 ARP 协议层), 并提取 ARP 请求 / 响应报文的硬件类型、协议类型、操作码、发送方 / 目标 IP 与 MAC 等字段;
    
    \item ARP 数据包解析: 基于给定 ARP 十六进制数据, 用 C++ 编写解析代码, 按指定格式输出 $9$ 个关键字段 (硬件类型、协议类型、硬件地址长度等).
\end{enumerate}

\section{实验步骤与结果分析}

\subsection{Wireshark 使用与 HTTP 协议分析}

\subsubsection{启动 Wireshark}

打开 Linux 终端, 输入命令 \mintinline{bash}|wireshark| 并回车, 启动 Wireshark 网络分析工具.

\subsubsection{选择捕获网卡}

在 Wireshark 顶部的网卡列表中, 选择当前设备使用的网卡 \mintinline{text}|eth0| (而非回环网卡 \mintinline{text}|lo|), 确保能捕获外网通信数据.

\begin{figure}[H]
    \centering
    \includegraphics[width=\textwidth]{./img/screenshot_0000.png}
\end{figure}

\subsubsection{启动浏览器与数据包捕获}

点击桌面 Chrome 浏览器图标打开浏览器, 随后点击 Wireshark 工具栏中的 ``捕获'' 按钮 (红色圆形图标), 开始数据包捕获, 界面底部显示 \mintinline{text}|Capturing from eth0|. 在 Chrome 地址栏输入 \mintinline{text}|http://www.nnu.edu.cn/|, 回车后等待页面完整加载.

\subsubsection{停止捕获与筛选 HTTP 报文}

页面加载完成后, 点击 Wireshark 的 ``停止捕获'' 按钮 (灰色方形图标), 结束抓包. 在 Wireshark 顶部的 \mintinline{text}|Apply a display filter| 输入框中输入 \mintinline{text}|http| 并回车, 报文列表仅显示 HTTP 协议相关报文.

\begin{figure}[H]
    \centering
    \includegraphics[width=\textwidth]{./img/screenshot_0001.png}
\end{figure}

\subsubsection{分析 Wireshark 主窗口结构}

Wireshark 主窗口分为三部分:

\begin{itemize}
    \item 数据包列表 (顶部): 显示报文序号 (\mintinline{text}|No.|)、捕获时间 (\mintinline{text}|Time|)、源 IP (\mintinline{text}|Source|)、目的 IP (\mintinline{text}|Destination|)、协议 (\mintinline{text}|Protocol|)、长度 (\mintinline{text}|Length|)、简要信息 (\mintinline{text}|Info|). 例如 No.3909 报文的 \mintinline{text}|Source| 为 \mintinline{text}|100.75.169.157|, \mintinline{text}|Destination| 为 \mintinline{text}|223.2.9.124|, \mintinline{text}|Info| 为 \mintinline{text}|GET / HTTP/1.1|.
    
    \item 数据包详细信息 (中部): 按分层协议展开单个报文内容, 从数据链路层 (\mintinline{text}|Ethernet II|) 到应用层 (\mintinline{text}|HTTP|).
    
    \item 数据包原始信息 (底部): 以十六进制 (左侧) 和 ASCII 码 (右侧) 显示报文原始字节, 如 \mintinline{text}|47 45 54| 对应 ASCII 的 \mintinline{text}|GET |.
\end{itemize}

\subsubsection{HTTP 报文分层字段分析}

以 No.3909 报文为例:

\begin{itemize}
    \item 数据链路层 (\mintinline{text}|Ethernet II|): 展开 \mintinline{text}|Ethernet II| 字段, 可见源 MAC 地址 (\mintinline{text}|Src|) 为 \mintinline{text}|6a:f7:3e:83:bb:78|, 目的 MAC 地址 (\mintinline{text}|Dst|) 为 \mintinline{text}|ee:ee:ee:ee:ee:ee| (广播地址), 类型 (\mintinline{text}|Type|) 为 \mintinline{text}|IPv4| (\mintinline{text}|0x0800|, 标识上层为 IP 协议).
    
    \begin{figure}[H]
        \centering
        \includegraphics[width=\textwidth]{./img/screenshot_0002.png}
    \end{figure}
    
    \item 网络层 (\mintinline{text}|IPv4|): 展开 \mintinline{text}|Internet Protocol Version 4| 字段, IP 版本 (\mintinline{text}|Version|) 为 \mintinline{text}|4|, 总长度 (\mintinline{text}|Total Length|) 为 \mintinline{text}|570 bytes|, 源 IP (\mintinline{text}|Src|) 为 \mintinline{text}|100.75.169.157|, 目的 IP (\mintinline{text}|Dst|) 为 \mintinline{text}|223.2.9.124|, 生存时间 (\mintinline{text}|Time to live|) 为 \mintinline{text}|64| (最多经过 $64$ 个路由器), 上层协议 (\mintinline{text}|Protocol|) 为 \mintinline{text}|TCP (6)| (标识上层为 TCP 协议), IPv4 头部的长度 (\mintinline{text}|Header Length|) 为 \mintinline{text}|20 bytes|.
    
    \begin{figure}[H]
        \centering
        \includegraphics[width=\textwidth]{./img/screenshot_0003.png}
    \end{figure}
    
    \item 传输层 (\mintinline{text}|TCP|): 展开 \mintinline{text}|Transmission Control Protocol| 字段, 源端口 (\mintinline{text}|Src Port|) 为 \mintinline{text}|48368| (客户端随机端口), 目的端口 (\mintinline{text}|Dst Port|) 为 \mintinline{text}|80| (HTTP 服务默认端口), 序列号 (\mintinline{text}|Seq|) 为 \mintinline{text}|1|, 确认号 (\mintinline{text}|Ack|) 为 \mintinline{text}|1|, 数据长度 (\mintinline{text}|Len|) 为 \mintinline{text}|518 bytes|, 标志 (\mintinline{text}|Flags|) 为 \mintinline{text}|0x018 (PSH, ACK)| (推送数据并确认), TCP 段头部的长度 (\mintinline{text}|Header Length|) 为 \mintinline{text}|32 bytes|.
    
    \begin{figure}[H]
        \centering
        \includegraphics[width=\textwidth]{./img/screenshot_0004.png}
    \end{figure}
    
    \item 应用层 (\mintinline{text}|HTTP|): 展开 \mintinline{text}|Hypertext Transfer Protocol| 字段, 请求方法为 \mintinline{text}|GET / HTTP/1.1| (获取网站根目录资源), 包含请求头 (\mintinline{text}|Accept|: 接受的资源类型、\mintinline{text}|Accept-Encoding|: 支持的压缩格式、\mintinline{text}|Accept-Language|: 语言偏好等), 使用的 HTTP 协议版本 (\mintinline{text}|Request Version|) 为 \mintinline{text}|HTTP/1.1|.
    
    \begin{figure}[H]
        \centering
        \includegraphics[width=\textwidth]{./img/screenshot_0005.png}
    \end{figure}
\end{itemize}

\subsection{ARP 协议分析}

\subsubsection{筛选 ARP 报文}

在 Wireshark 筛选框中输入 \mintinline{text}|arp| 并回车, 报文列表仅显示 ARP 协议报文, 包含 ARP 请求 (\mintinline{text}|Who has ...? Tell ...”|) 和 ARP 响应 (\mintinline{text}|... is at ...|) 两类.

\begin{figure}[H]
    \centering
    \includegraphics[width=\textwidth]{./img/screenshot_0006.png}
\end{figure}

\subsubsection{ARP 报文分层结构分析}

查看任意 ARP 报文的详细信息, 发现仅包含 \mintinline{text}|Ethernet II| (数据链路层) 和 \mintinline{text}|Address Resolution Protocol| (ARP 协议层), 无传输层和应用层——因 ARP 工作在网络层与数据链路层之间, 负责 IP 与 MAC 的映射, 不属于传统四层模型.

\subsubsection{ARP 协议字段分析}

以 No.1533 ARP 请求报文为例, 展开 \mintinline{text}|Address Resolution Protocol (request)| 字段, 各关键字段如下:

\begin{figure}[H]
    \centering
    \includegraphics[width=\textwidth]{./img/screenshot_0007.png}
\end{figure}

\begin{itemize}
    \item 硬件类型 (\mintinline{text}|Hardware type|): \mintinline{text}|Ethernet (1)| (表示底层为以太网);
    
    \item 协议类型 (\mintinline{text}|Protocol type|): \mintinline{text}|IPv4 (0x0800)| (表示映射的上层协议为 IPv4);
    
    \item 硬件地址长度 (\mintinline{text}|Hardware size|): \mintinline{text}|6 bytes| (MAC 地址长度固定为 6 字节);
    
    \item 协议地址长度 (\mintinline{text}|Protocol size|): \mintinline{text}|4 bytes| (IPv4 地址长度固定为 4 字节);
    
    \item 操作码 (\mintinline{text}|Opcode|): \mintinline{text}|request (1)| (标识为 ARP 请求);
    
    \item 发送方 MAC (\mintinline{text}|Sender MAC address|): \mintinline{text}|ee:ee:ee:ee:ee:ee| (请求方 MAC);
    
    \item 发送方 IP (\mintinline{text}|Sender IP address|): \mintinline{text}|172.21.229.13| (请求方 IP);
    
    \item 目标 MAC (\mintinline{text}|Target MAC address|): \mintinline{text}|00:00:00:00:00:00| (请求时未知目标 MAC, 填充为全 0);
    
    \item 目标 IP (\mintinline{text}|Target IP address|): \mintinline{text}|100.75.169.157| (需解析 MAC 的目标 IP).
\end{itemize}

\subsubsection{ARP 响应报文对比}

ARP 响应报文的操作码 (\mintinline{text}|Opcode|) 为 \mintinline{text}|reply (2)|, 目标 MAC 地址更新为 \mintinline{text}|6a:f7:3e:83:bb:78| (已知目标 MAC), 其他字段与请求报文一一对应, 实现 ``IP$\rightarrow$MAC'' 的映射告知.

\begin{figure}[H]
    \centering
    \includegraphics[width=\textwidth]{./img/screenshot_0008.png}
\end{figure}

\subsection{ARP 数据包解析}

\subsubsection{明确解析需求}

给定 ARP 十六进制数据: \mintinline{text}|0001 0800 0604 0001 eeee eeee eeee ac15 e50b 0000 0000 0000 6463 ae47|, 需解析为 $9$ 个字段, 输出格式为 ``\mintinline{text}|Hardware Type: ...|''、``\mintinline{text}|Protocol Type: ...|'' 等. 报文总长 $28$ 字节, 分为两部分:

\begin{itemize}
    \item 前 $8$ 字节是 ARP 报头,包含硬件类型、上层协议类型、MAC 地址长度、IP 地址长度、操作类型, 用来规定网络类型与报文功能;
    
    \item 后 $20$ 字节是地址信息段, 包含源 MAC 地址、源 IP 地址、目的 MAC 地址、目的 IP 地址, 用于明确通信双方的 IP 与 MAC 映射关系.
\end{itemize}

对该十六进制数据按各字段的字节长度拆分后, 即可依次解析出这 $9$ 个字段的具体内容.

\begin{figure}[H]
    \centering
    \includegraphics[width=\textwidth]{./img/arp_message.png}
\end{figure}

\subsubsection{编写 C++ 解析代码}

将十六进制字符串转为字节数组, 按 ARP 协议字段的固定偏移和长度提取数据, 再转换为人类可读格式 (如 MAC 转冒号分隔、IP 转点分十进制). 代码需补充部分如下:

\begin{minted}{C++}
void parseARPPacket(const uint8_t *arpData, const std::string &outputFile) 
{
    ...
    
    uint16_t hardwareType = ntohs(*(const uint16_t *)&arpData[0]);
    uint16_t protocolType = ntohs(*(const uint16_t *)&arpData[2]);
    uint8_t hardwareLen = arpData[4];
    uint8_t protocolLen = arpData[5];
    uint16_t operation = ntohs(*(const uint16_t *)&arpData[6]);
    
    char senderMacStr[18];
    sprintf(senderMacStr, "%02x:%02x:%02x:%02x:%02x:%02x", arpData[8], arpData[9], arpData[10], arpData[11], arpData[12], arpData[13]);
    
    char senderIpStr[16];
    sprintf(senderIpStr, "%d.%d.%d.%d", arpData[14], arpData[15], arpData[16], arpData[17]);
    
    char targetMacStr[18];
    sprintf(targetMacStr, "%02x:%02x:%02x:%02x:%02x:%02x", arpData[18], arpData[19], arpData[20], arpData[21], arpData[22], arpData[23]);
    
    char targetIpStr[16];
    sprintf(targetIpStr, "%d.%d.%d.%d", arpData[24], arpData[25], arpData[26], arpData[27]);
    
    outFile << "Hardware Type: " << hardwareType << std::endl;
    outFile << "Protocol Type: 0x" << std::hex << std::setw(4) << std::setfill('0') << protocolType << std::dec << std::endl;
    outFile << "Hardware Address Length: " << static_cast<int>(hardwareLen) << std::endl;
    outFile << "Protocol Address Length: " << static_cast<int>(protocolLen) << std::endl;
    outFile << "Operation: " << operation << std::endl;
    outFile << "Sender MAC: " << senderMacStr << std::endl;
    outFile << "Sender IP: " << senderIpStr << std::endl;
    outFile << "Target MAC: " << targetMacStr << std::endl;
    outFile << "Target IP: " << targetIpStr << std::endl;
    
    outFile.close();
}
\end{minted}

\subsubsection{编译与运行代码}

在 Linux 终端中执行编译命令: \mintinline{bash}|g++ arp_parse.cpp -o arp_parse| (无报错则生成可执行文件); 执行运行命令: \mintinline{bash}|./arp_parse 0001...ae47 result.txt|, 输出文件 \mintinline{text}|result.txt| 结果如下:

\begin{figure}[H]
    \centering
    \includegraphics[width=\textwidth]{./img/screenshot_0009.png}
\end{figure}

结果说明, 发送方 IP 由字节 \mintinline{text}|0xac (172)|、\mintinline{text}|0x15 (21)|、\mintinline{text}|0xe5 (229)|、\mintinline{text}|0x0b (11)| 转换, 目标 IP 由 \mintinline{text}|0x64 (100)|、\mintinline{text}|0x63 (99)|、\mintinline{text}|0xae (174)|、\mintinline{text}|0x47 (71)| 转换, 符合 ARP 请求报文特征 —— 目标 MAC 为 \mintinline{text}|00:00:00:00:00:00|.

\section{实验中遇到的问题及体会}

在实验过程中, 曾遇到 ARP 报文长度不一致的问题: 筛选出的 ARP 报文长度呈现 $42$ 字节与 $60$ 字节两种情况. 起初并不清楚差异的成因, 经过分析后发现, ARP 报文的最小长度为 $42$ 字节, 这是由数据链路层头部 $14$ 字节与 ARP 报文本身 $28$ 字节共同构成的; 而 $60$ 字节的长度则与以太网最小帧长度相关——以太网最小帧长度为 $64$ 字节 (包含前导码和帧间隙), 当数据部分不足 $60$ 字节时, 会自动添加填充字节, 因此部分 ARP 报文会显示为 $60$ 字节.

通过这次实验, 对计算机网络的认知有了多方面深化. 课本中 ``网络协议分层封装'' 的理论, 在 Wireshark 抓包过程中变得直观: HTTP 请求需经 TCP (传输层)、IP (网络层)、Ethernet (数据链路层) 依次封装, 每层都会添加头部信息, 接收方则反向解封装, 这让我切实理解了 ``分层解耦、各司其职'' 的设计思想.

此前仅知晓 ``IP 用于路由、MAC 用于局域网通信'', 但 ARP 抓包让我看到其 ``底层支撑'' 作用: 若没有 ARP 将目标 IP 解析为 MAC 地址, 即便 IP 地址正确, 数据也无法在局域网内传输; 而 ARP 的 ``广播请求、单播响应'' 机制, 更清晰解释了 ``同一网段设备能直接通信'' 的底层原理.

同时也意识到协议规范的核心意义: 解析 ARP 数据包时, 只要严格依照 ``硬件类型 2 字节、协议类型 2 字节、硬件地址长度 1 字节'' 的固定格式提取字段, 就能正确解析; 反之, 任何字段偏移或长度错误, 都会导致解析失败, 这印证了网络设备互通对严谨协议规范的依赖.

此外, 问题排查的过程也提升了实践能力. 从对 ARP 报文长度差异的疑问, 到结合以太网帧长度要求、协议格式进行理论分析, 再通过抓包数据对比验证, 整个 ``发现问题——分析原因——解决问题'' 的流程, 比单纯执行实验步骤更能加深对知识的掌握, 也让我更熟悉网络实验的调试思路.

\end{document}