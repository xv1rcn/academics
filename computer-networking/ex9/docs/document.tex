\documentclass[12pt, a4paper]{ctexart}
\usepackage[margin=2cm]{geometry}
\usepackage{libertine}

\usepackage{titlesec}
\usepackage{zhnumber}
\titleformat*{\section}{\Large\bfseries\raggedright}
\renewcommand{\thesection}{\zhnum{section}}
\renewcommand{\thesubsection}{\arabic{subsection}}

\setcounter{secnumdepth}{4}
\renewcommand{\theparagraph}{\thesubsubsection.\arabic{paragraph}}
\renewcommand{\paragraph}[1]{
    \refstepcounter{paragraph}
    {\bfseries\theparagraph\quad#1}\par
    \vspace{2pt}
    \noindent
}

\usepackage{enumitem}
\setlist[enumerate]{itemsep=2pt, parsep=0pt, partopsep=0pt, topsep=2pt}
\setlist[itemize]{itemsep=2pt, parsep=0pt, partopsep=0pt, topsep=2pt}
\linespread{1.2}

\usepackage{graphicx}

\usepackage{minted}
\setminted{
    breaklines=true, escapeinside=||, fontsize=\small, frame=lines,
    mathescape=$$, numbers=none, style=bw, tabsize=4
}

\usepackage{siunitx}

\begin{document}
    
\pagestyle{plain}
\thispagestyle{empty}

\noindent
\begin{tabular*}{\textwidth}{l @{\extracolsep{\fill}} r @{\extracolsep{6pt}} l}
    \LARGE{\textbf{实验报告}} & 计算机网络 & \textit{Computer Networking} \\
\end{tabular*}\\\\
\begin{tabular*}{\textwidth}{l l}
    \textbf{报告标题: } & \textbf{路由器 DHCP 与 NAT 配置} \\
    \textbf{学号: } & 19240212 \\
    \textbf{姓名: } & 华博文 \\
    \textbf{日期: } & 2025 年 11 月 17 日 \\
\end{tabular*}\\
\rule[2ex]{\textwidth}{2pt}

\section{实验目的}

本实验旨在掌握路由器上 DHCP 服务器的配置方法,实现多网段主机的 IP 地址自动分配(含排除地址、默认网关、DNS 服务器配置);同时掌握动态路由协议 RIP 的配置方法,理解多 VLAN 环境下子接口路由与动态路由的协同工作机制;通过搭建复杂网络拓扑、完成设备配置及连通性测试,实现跨 VLAN、跨路由器的全网通信,强化对 DHCP 服务与动态路由在网络中自动化管理作用的认知。

\section{实验内容简要描述}

\begin{enumerate}
    \item \textbf{DHCP 服务配置}:在路由器上创建多个 DHCP 地址池,为不同网段的主机自动分配 IP 地址,配置排除地址、默认网关、DNS 服务器等参数,验证 IP 地址分配结果;

    \item \textbf{路由器动态路由(RIP)配置}:搭建多 VLAN、多网段网络拓扑,完成交换机 VLAN 划分与 Trunk 链路配置,路由器子接口配置及 RIP 协议启用,实现全网动态路由连通性;

    \item \textbf{DHCP 与动态路由协同验证}:验证 DHCP 分配的主机在动态路由网络中的跨网段通信能力,分析 DHCP 服务与动态路由在网络架构中的角色分工;
\end{enumerate}

\section{实验步骤与结果分析}

\subsection{DHCP 服务配置与验证}

\subsubsection{搭建 DHCP 实验网络拓扑}

在 Packet Tracer 中拖拽 2 台路由器(Router0、Router1)、2 台交换机(Switch0、Switch1)、4 台主机(PC0、PC1、PC2、PC3),按如下逻辑连接设备:

\begin{enumerate}
    \item Router0 的 FastEthernet0/0 接口连接 Switch0,FastEthernet0/1 接口连接 Switch1,Serial0/1/0 接口与 Router1 的 Serial0/1/0 接口互联;

    \item Switch0 的 FastEthernet0/2、0/3 端口分别连接 PC0、PC1;Switch1 的 FastEthernet0/2、0/3 端口分别连接 PC2、PC3;
\end{enumerate}

完成拓扑搭建后,呈现设备间的物理连接关系。

\begin{figure}[H]
    \centering
    \includegraphics[width=\textwidth]{./img/screenshot_0000.png}
\end{figure}

\subsubsection{配置路由器接口 IP 及 DHCP 地址池}

进入 Router0 和 Router1 的全局配置模式,配置各接口 IP,为 DHCP 服务和路由通信提供基础。在 Router0 上配置两个 DHCP 地址池,分别对应 \mintinline{text}`192.168.3.0/24` 和 \mintinline{text}`192.168.4.0/24` 网段。配置完成后,路由器将为连接的 PC 自动分配指定网段内的 IP 地址,排除保留地址范围。

\begin{figure}[H]
    \centering
    \includegraphics[width=\textwidth]{./img/screenshot_0001.png}
\end{figure}

\begin{figure}[H]
    \centering
    \includegraphics[width=\textwidth]{./img/screenshot_0002.png}
\end{figure}

\subsubsection{验证 PC 的 DHCP 地址获取}

查看 PC0、PC1、PC2、PC3 的 IP 配置:

\begin{itemize}
    \item PC0、PC1 获取的 IP 地址在 \mintinline{text}`192.168.3.12` ~ \mintinline{text}`192.168.3.254` 之间;
    
    \item PC2、PC3 获取的 IP 地址在 \mintinline{text}`192.168.4.21` ~ \mintinline{text}`192.168.4.254` 之间。
\end{itemize}

\begin{figure}[H]
    \centering
    \includegraphics[width=\textwidth]{./img/screenshot_0003.png}
\end{figure}

\begin{figure}[H]
    \centering
    \includegraphics[width=\textwidth]{./img/screenshot_0004.png}
\end{figure}

\begin{figure}[H]
    \centering
    \includegraphics[width=\textwidth]{./img/screenshot_0005.png}
\end{figure}

\begin{figure}[H]
    \centering
    \includegraphics[width=\textwidth]{./img/screenshot_0006.png}
\end{figure}

PC 成功获取到对应网段的 IP、默认网关和 DNS 服务器,说明 DHCP 配置生效。

\subsection{路由器动态路由(RIP)配置与验证}

\subsubsection{搭建多 VLAN 动态路由实验拓扑}

调整网络拓扑,包含 2 台交换机(Switch0、Switch1)、2 台路由器(Router0、Router1)、4 台主机(PC0、PC1、PC2、PC3),连接方式如下:

\begin{itemize}
    \item Switch0 的 FastEthernet0/2、0/3 端口分别连接 PC0、PC1,FastEthernet0/1 端口以 Trunk 模式连接 Router0 的 FastEthernet0/0 接口;
    
    \item Switch1 的 Switch1 的 FastEthernet0/2、0/3 端口分别连接 PC2、PC3,FastEthernet0/1 端口以 Trunk 模式连接 Router1 的 FastEthernet0/0 接口;
    
    \item Router0 与 Router1 通过 Serial0/1/0 接口互联;
\end{itemize}

完成多 VLAN、多网段的拓扑扩展。

\begin{figure}[H]
    \centering
    \includegraphics[width=\textwidth]{./img/screenshot_0007.png}
\end{figure}

\subsubsection{配置交换机 VLAN 与 Trunk 链路}

配置 Switch0,创建 VLAN2、VLAN3,配置接入端口与 Trunk 链路:

\begin{figure}[H]
    \centering
    \includegraphics[width=\textwidth]{./img/screenshot_0008.png}
\end{figure}

配置 Switch1,创建 VLAN4、VLAN5,配置接入端口与 Trunk 链路:

\begin{figure}[H]
    \centering
    \includegraphics[width=\textwidth]{./img/screenshot_0009.png}
\end{figure}

此步骤实现交换机对多 VLAN 的支持及 Trunk 链路的配置,为路由器子接口路由奠定基础。

\subsubsection{配置路由器子接口与 RIP 协议}

进入 Router0 配置子接口与 RIP 协议:

\begin{figure}[H]
    \centering
    \includegraphics[width=\textwidth]{./img/screenshot_0010.png}
\end{figure}

进入 Router1 配置子接口与 RIP 协议:

\begin{figure}[H]
    \centering
    \includegraphics[width=\textwidth]{./img/screenshot_0011.png}
\end{figure}

通过子接口配置实现 VLAN 间路由,通过 RIP 协议实现动态路由信息交换。

\subsubsection{动态路由网络连通性测试}

在 PC0 的命令行界面,执行如下 ping 命令:

\begin{itemize}
    \item 执行 \mintinline{text}`ping 192.168.5.1` 命令,验证数据包能否通过 RIP 动态路由到达 Router1 的子接口;
    
    \item 执行 \mintinline{text}`ping 192.168.4.11` 命令,验证数据包能否通过多网段、多 VLAN 转发到达 PC2;
\end{itemize}

若返回连续 “Reply” 信息,说明动态路由配置生效,全网连通性正常。

\section{实验中遇到的问题及体会}

在 DHCP 配置阶段,曾因地址池的网络掩码配置错误,导致 PC 获取的 IP 地址与预期网段不符。排查时通过对比路由器 DHCP 配置命令与 PC 实际获取的 IP 信息,发现是地址池的 \mintinline{text}`network` 命令中子网掩码设置失误,修正后 PC 成功获取到正确网段的 IP。这让我意识到,DHCP 配置中地址池的网络参数必须与接口网段严格匹配,任何参数偏差都会导致 IP 分配失败。

在动态路由配置时,因 Switch0 的 Trunk 链路未配置 \mintinline{text}`switchport trunk allowed vlan all` 命令,导致 VLAN2、VLAN3 的数据包无法通过 Trunk 链路到达 Router0,进而造成跨 VLAN 通信中断。通过在交换机接口下查看 Trunk 允许的 VLAN 列表,补充该命令后问题解决。此过程让我深刻理解,Trunk 链路的 VLAN 允许列表是多 VLAN 通信的关键,必须确保所需 VLAN 的通行权限。

通过本次实验,我对 DHCP 服务与动态路由的协同工作有了清晰认知:DHCP 解决了主机 IP 的自动化分配问题,减少了人工配置的工作量与失误率;而动态路由 RIP 则在多网段、多 VLAN 的复杂拓扑中,实现了路由信息的自动交换与网络收敛,两者结合大幅提升了网络管理的自动化程度。

此外,实验强化了我对 “分层网络配置” 的理解。从 DHCP 的应用层服务,到动态路由的网络层协议,再到 VLAN 与 Trunk 的数据链路层配置,每一层的技术都在各自维度支撑网络功能,且层与层之间通过协议封装紧密协作。这种分层设计不仅让网络功能模块化,也为故障排查提供了清晰的逻辑路径 —— 可从应用层的服务现象(如 IP 获取失败)逐步向下层(网络层、数据链路层)追溯原因。

最后,Packet Tracer 的虚拟环境为实验提供了高效的验证平台。在虚拟环境中,我可以大胆尝试不同的配置组合,快速验证技术原理,这种 “试错 — 验证 — 总结” 的学习模式,极大地提升了我的网络配置技能与故障排查思维,为后续深入学习网络技术奠定了坚实的实践基础。

\end{document}