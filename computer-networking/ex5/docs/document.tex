\documentclass[12pt, a4paper]{ctexart}
\usepackage[margin=2cm]{geometry}
\usepackage{libertine}

\usepackage{titlesec}
\usepackage{zhnumber}
\titleformat*{\section}{\Large\bfseries\raggedright}
\renewcommand{\thesection}{\zhnum{section}}
\renewcommand{\thesubsection}{\arabic{subsection}}

\setcounter{secnumdepth}{4}
\renewcommand{\theparagraph}{\thesubsubsection.\arabic{paragraph}}
\renewcommand{\paragraph}[1]{
    \refstepcounter{paragraph}
    {\bfseries\theparagraph\quad#1}\par
    \vspace{2pt}
    \noindent
}

\usepackage{enumitem}
\setlist[enumerate]{itemsep=2pt, parsep=0pt, partopsep=0pt, topsep=2pt}
\setlist[itemize]{itemsep=2pt, parsep=0pt, partopsep=0pt, topsep=2pt}
\linespread{1.2}

\usepackage{graphicx}

\usepackage{minted}
\setminted{
    breaklines=true, escapeinside=||, fontsize=\small, frame=lines,
    mathescape=$$, numbers=none, style=bw, tabsize=4
}

\usepackage{siunitx}

\begin{document}
    
\pagestyle{plain}
\thispagestyle{empty}

\noindent
\begin{tabular*}{\textwidth}{l @{\extracolsep{\fill}} r @{\extracolsep{6pt}} l}
    \LARGE{\textbf{实验报告}} & 计算机网络 & \textit{Computer Networking} \\
\end{tabular*}\\\\
\begin{tabular*}{\textwidth}{l l}
    \textbf{报告标题: } & \textbf{协议分析——IP 协议} \\
    \textbf{学号: } & 19240212 \\
    \textbf{姓名: } & 华博文 \\
    \textbf{日期: } & 2025 年 10 月 5 日 \\
\end{tabular*}\\
\rule[2ex]{\textwidth}{2pt}

\section{实验目的}

本实验旨在掌握 Wireshark 的启动、网卡选择、IP 相关数据包捕获与筛选操作, 熟悉其主窗口三部分结构及功能; 理解 IPv4 协议头部的完整结构 (字段组成、含义及格式), 能准确提取 Version、Header Length、TTL、Protocol、源 / 目的 IP 等关键字段信息; 掌握 IP 分片的触发条件、分片原理 (标志位、分片偏移) 及接收端重组过程, 理解 ICMP 协议与 IP 协议的关联; 学会按 IP 协议格式解析指定十六进制 IP 数据报, 提取 IPv4 头部、ICMP 头部及载荷内容并规范输出; 最终加深对计算机网络分层模型 (数据链路层、网络层) 的理解, 验证 IP 协议在数据传输中的封装与分片机制.

\section{实验内容简要描述}

\begin{enumerate}
    \item Wireshark 与 IP 头部及分片分析: 通过终端启动 Wireshark, 执行 \mintinline{bash}|ping www.nnu.edu.cn -s 4500 -c 1| 命令, 启动抓包后等待命令执行完成, 停止抓包并筛选 IP 报文, 分析 IP 头部关键字段及 IP 分片的数量、标志位、重组逻辑, 同时提取 ICMP 头部字段;
    
    \item IP 数据报解析: 基于给定 IP 十六进制数据, 用 C++ 编写解析逻辑, 按指定格式输出 IPv4 头部、ICMP 头部及载荷内容, 并将结果保存至指定文件.
\end{enumerate}

\section{实验步骤与结果分析}

\subsection{Wireshark 与 IP 头部及分片分析}

\subsubsection{启动 Wireshark}

打开 Linux 终端, 输入命令 \mintinline{bash}|wireshark| 并回车, 启动 Wireshark 网络分析工具.

\subsubsection{选择捕获网卡}

在 Wireshark 顶部的网卡列表中, 选择当前设备使用的网卡 \mintinline{text}|eth0| (而非回环网卡 \mintinline{text}|lo|), 确保能捕获外网通信数据.

\subsubsection{执行 \mintinline{bash}|ping| 命令}

打开另一个 Linux 终端, 输入命令 \mintinline{bash}|ping www.nnu.edu.cn -s 4500 -c 1| 并回车: 其中 \mintinline{bash}|-s 4500| 指定 ICMP 报文数据部分大小为 4500 字节, \mintinline{bash}|-c 1| 表示仅发送 1 次请求; 终端反馈显示 \mintinline{text}|4508 bytes from 223.2.9.124: icmp_seq=1 ttl=59 time=1.62 ms|, 表明 ping 请求已发送并收到响应.

\begin{figure}[H]
    \centering
    \includegraphics[width=\textwidth]{./img/screenshot_0001.png}
\end{figure}

\subsubsection{停止捕获与筛选 IP 报文}

待 ping 命令执行完成, 点击 Wireshark 的 ``停止捕获'' 按钮 (灰色方形图标), 结束抓包. 在 在 Wireshark 顶部的 \mintinline{text}|Apply a display filter| 输入框中输入 \mintinline{text}|ip.dst==223.2.9.124| 并回车, 报文列表仅显示目的 IP 为该地址的 IP 相关报文.

\begin{figure}[H]
    \centering
    \includegraphics[width=\textwidth]{./img/screenshot_0002.png}
\end{figure}

\subsubsection{IP 头部字段分析}

以 No.1866 报文 (IPv4 分片) 为例, 展开 ``Internet Protocol Version 4'' 字段, 各关键字段如下:

\begin{figure}[H]
    \centering
    \includegraphics[width=\textwidth]{./img/screenshot_0003.png}
\end{figure}

\begin{itemize}
    \item \textbf{Version} (版本): \mintinline{text}|4|, 标识为 IPv4 协议;
    
    \item \textbf{Header Length} (头部长度): \mintinline{text}|20 bytes| (IPv4 头部长度以 4 字节为单位);
    
    \item \textbf{Total Length} (总长度): \mintinline{text}|1476 bytes| (标识整个 IP 数据报的长度, 含头部和数据);
    
    \item \textbf{Identification} (标识): \mintinline{text}|0xb68b|, 十进制为 46731, 用于标识同一原始 IP 数据报的所有分片;
    
    \item \textbf{Flags} (标志): \mintinline{text}|0x2000|, 其中 ``More fragments'' 位设为 \mintinline{text}|1|, 标识当前为分片且后续仍有分片, ``Don't fragment'' 位设为 0, 允许分片;
    
    \item \textbf{Fragment offset} (分片偏移): \mintinline{text}|0|, 标识当前分片在原始数据报数据部分的起始位置 (以 8 字节为单位);
    
    \item \textbf{TTL} (生存时间): \mintinline{text}|64|, 表示数据报最多可经过 64 个路由器, 每经过一个路由器值减 1;
    
    \item \textbf{Protocol} (上层协议): \mintinline{text}|1|, 标识上层为 ICMP 协议 (标准协议号中 1 对应 ICMP);
    
    \item \textbf{Source IP} (源 IP 地址): \mintinline{text}|100.99.174.97|, 为发送 ping 请求的本地 IP;
    
    \item \textbf{Destination IP} (目的 IP 地址): \mintinline{text}|223.2.9.124|, 为南京师范大学官网相关 IP.
\end{itemize}

\subsubsection{IP 分片与重组分析}

\begin{enumerate}
    \item \textbf{分片数量与特征}: 筛选后共显示 4 个报文, 其中 3 个为 IPv4 分片 (No.1866、1867、1868), 1 个为完整 ICMP 报文 (No.1869). 3 个分片的 Protocol 均为 IPv4, Info 均包含 ``Fragmented IP protocol'', 且 Flags 字段的 ``More fragments'' 均设为 1; 分片偏移依次为0、1456、2912 (十进制), 对应数据部分起始位置, 每个分片的 Length 均为 1490 bytes (含数据链路层头部).
    
    \begin{figure}[H]
        \centering
        \includegraphics[width=\textwidth]{./img/screenshot_0004.png}
    \end{figure}
    
    \begin{figure}[H]
        \centering
        \includegraphics[width=\textwidth]{./img/screenshot_0005.png}
    \end{figure}
    
    \item \textbf{分片重组逻辑}: 在 3 个 IPv4 分片的详细信息中, 均显示 \mintinline{text}|Reassembled IPv4 in frame: 1869|, 表明 Wireshark 已将这 3 个分片在 No.1869 报文中重组为完整的原始 IP 数据报.
    
    \item \textbf{ICMP头部与数据长度分析}: 展开 No.1869 报文的 ``Internet Control Message Protocol'' 字段, ICMP 头部信息如下:
        
    \begin{itemize}
        \item \textbf{Type} (类型): \mintinline{text}|8|, 标识为 Echo (ping) 请求;
            
        \item \textbf{Code} (代码): \mintinline{text}|0|, 表示无附加含义;
            
        \item \textbf{Sequence number} (序列号): \mintinline{text}|1|, 与 \mintinline{bash}|-c 1| 参数对应;
            
        \item \textbf{Data} (数据部分长度): \mintinline{text}|4492 bytes|, 与指定的 4500 bytes 存在差异, 原因是 ICMP 头部占用 8 bytes (Type、Code、Checksum、Identifier、Sequence number).
    \end{itemize}
    
    \begin{figure}[H]
        \centering
        \includegraphics[width=\textwidth]{./img/screenshot_0006.png}
    \end{figure}
\end{enumerate}

\subsection{IP 数据报解析}

\subsubsection{明确解析需求}

给定 IP 十六进制数据: \mintinline{text}|4500 002d 1234 0000 4001 1943 6455 77e8 df02 007b 0000 5fec 0000 0000 4920 4c6f 7665 204e 4e55 21|, 需按以下规则解析并输出, 结果保存至 \mintinline{text}|/home/headless/result.txt|:

\begin{itemize}
    \item \textbf{解析范围}: 分为 IPv4 头部、ICMP 头部、Payload 三部分, IPv4 头部固定 20 bytes (对应前 40 个十六进制字符), 后续为 ICMP 头部 (8 bytes, 对应 16 个十六进制字符) 及 Payload;
    
    \item \textbf{输出格式}: 需包含 ``IPv4 Header'' ``ICMP Header'' ``Payload'' 三大模块, 共 12 项字段, 其中 Version、Header Length、TTL、Protocol、Type、Code 为十进制值, 源 / 目的 IP 为点分十进制, Payload 为 ASCII 字符串, 字段与值用英文冒号分隔.
\end{itemize}

\subsubsection{编写 C++ 解析代码}

代码需补充部分如下:

\begin{minted}{C++}
int main(int argc, char **argv) {
    const std::string hexIpData =
    "4500002d1234000040011943645577e8df02097b08005f"
    "ec0000000049204c6f7665204e4e5521";
    
    std::vector<uint8_t> ipBytes;
    for (size_t i = 0; i < hexIpData.size(); i += 2) {
        std::string byteStr = hexIpData.substr(i, 2);
        ipBytes.push_back(
        static_cast<uint8_t>(std::stoi(byteStr, nullptr, 16)));
    }
    
    const uint8_t version = (ipBytes[0] >> 4) & 0x0F;
    const uint8_t ihl = ipBytes[0] & 0x0F;
    const uint16_t headerLength = ihl * 4;
    const uint8_t ttl = ipBytes[8];
    const uint8_t protocol = ipBytes[9];
    
    struct in_addr srcIp;
    srcIp.s_addr = *reinterpret_cast<const uint32_t *>(&ipBytes[12]);
    const std::string srcIpStr = inet_ntoa(srcIp);
    
    struct in_addr dstIp;
    dstIp.s_addr = *reinterpret_cast<const uint32_t *>(&ipBytes[16]);
    const std::string dstIpStr = inet_ntoa(dstIp);
    
    const size_t icmpStart = headerLength;
    const uint8_t icmpType = ipBytes[icmpStart];
    const uint8_t icmpCode = ipBytes[icmpStart + 1];
    
    const size_t payloadStart = icmpStart + sizeof(struct icmphdr);
    std::string payloadContent;
    for (size_t i = payloadStart; i < ipBytes.size(); ++i) {
        payloadContent += static_cast<char>(ipBytes[i]);
    }
    
    std::ofstream fout("result.txt");
    if (!fout.is_open()) {
        return 1;
    }
    
    fout << "=== IPv4 Header ===" << std::endl;
    fout << "Version: " << static_cast<int>(version) << std::endl;
    fout << "Header Length: " << headerLength << std::endl;
    fout << "TTL: " << static_cast<int>(ttl) << std::endl;
    fout << "Protocol: " << static_cast<int>(protocol) << std::endl;
    fout << "Source IP: " << srcIpStr << std::endl;
    fout << "Dest IP: " << dstIpStr << std::endl;
    
    fout << "=== ICMP Header ===" << std::endl;
    fout << "Type: " << static_cast<int>(icmpType) << std::endl;
    fout << "Code: " << static_cast<int>(icmpCode) << std::endl;
    
    fout << "=== Payload ===" << std::endl;
    fout << "Content: " << payloadContent << std::endl;
    
    fout.close();
    return 0;
}
\end{minted}

\subsubsection{编译与运行代码}

在 Linux 终端中执行编译命令: \mintinline{bash}|g++ ip_parse.cpp -o ip_parse| (无报错则生成可执行文件); 执行运行命令: \mintinline{bash}|./ip_parse|, 输出文件 \mintinline{text}|result.txt| 结果如下.

\begin{minted}{text}
=== IPv4 Header ===
Version: 4
Header Length: 20
TTL: 64
Protocol: 1
Source IP: 100.85.119.232
Dest IP: 223.2.9.123
=== ICMP Header ===
Type: 8
Code: 0
=== Payload ===
Content: I Love NNU!
\end{minted}

\section{实验中遇到的问题及体会}

实验初期, 在对 IPv4 分片的观察中发现了长度差异的疑问: 筛选出的所有 IPv4 分片长度均为 1490 bytes, 而原始 ping 请求中明确指定数据部分为 4500 bytes, 这使得分片数量与长度之间的对应关系难以理解. 经过深入分析后得知, 以太网的 MTU 通常为 1500 bytes, IP 数据报在传输时需满足 ``IP头部 (20 bytes) $+$ 数据部分 $\leq$ 1500 bytes'' 的条件, 据此推算每个分片的数据部分最大应为 1480 bytes, 但实际通过抓包工具观察到的数据部分却为 1456 bytes (这一数值与分片偏移以 8 字节为单位的规则相符). 进一步计算发现, 3 个分片的数据部分总和加上 ICMP 头部的长度为  4376 bytes, 与预期的 4500 bytes 仍存在差异, 继续排查后才找到原因——数据链路层头部 (14 bytes) 占用了部分长度, 因此在计算时需要结合帧的整体长度进行综合考量. 

在解析指定十六进制 IP 数据报的过程中, 曾出现字段偏移错误的问题: 初期误将 ICMP 头部的 Type 字段偏移定位到第 43、44 个字符, 直接导致了解析结果的错误. 为修正这一问题, 团队对照了 IPv4 头部的规范——IPv4 头部固定长度为 20 bytes (对应 40 个十六进制字符), 据此重新确认 ICMP 头部应起始于第 41 个字符, 调整偏移位置后, 再次进行解析得到了正确的结果.

通过此次实验, 对网络分层模型的认知得到了深化. 课本中 ``IP 协议工作于网络层, 负责路由与分片'' 的理论知识, 在实验操作中变得更加直观具体: IP 数据报无法直接传输, 必须封装在数据链路层的帧结构中, 且会因受 MTU 限制而进行分片;接收端则需要通过 IP 数据报中的标识 (Identification) 字段和分片偏移字段, 将多个分片重新组合成完整的数据报, 这一整个过程清晰地体现了网络模型 ``分层解耦、各司其职'' 的设计思想. 同时, 也进一步理解了 ICMP 协议作为网络层辅助协议的特性——它本身不具备独立传输能力, 必须依赖 IP 协议进行封装后才能完成数据传输, 两者之间的关联关系不再是抽象的概念.

实验同样让我们认识到协议规范的核心意义. 在解析 IP 数据报的过程中, 严格遵循各项协议规范是确保解析成功的关键, 例如 Version 字段占 4 位、Header Length 字段以 4 字节为单位、分片偏移字段以 8 字节为单位等规则, 任何一项的忽视都可能导致解析失败. 以 Header Length 字段为例, 若误将其单位当作 1 字节来计算, 会直接导致后续所有字段的偏移位置全部出错, 无法正确读取数据. 这一经历也印证了网络设备之间能够实现互通的基础, 正是对严谨协议规范的共同遵循, 任何字段在格式或长度上的微小偏差, 都可能导致数据解析的彻底失败.

此外, 问题排查能力在实验中得到了显著提升. 从最初对 IP 分片长度差异的疑惑, 到结合 MTU 数值、IP 头部长度进行理论计算, 再通过 Wireshark 工具查看 ``Reassembled IPv4 in frame'' 选项验证分片重组逻辑; 从解析 IP 数据报时出现的字段偏移错误, 到对照 IPv4 协议格式重新修正字段定位, 整个过程遵循 ``发现问题—理论分析—验证解决'' 的逻辑闭环. 这种主动排查问题的过程, 比单纯按照步骤执行实验更能加深对知识的理解和掌握, 同时也让我们熟悉了 ``结合协议规范 $+$ 关注工具细节'' 的网络实验调试思路, 为后续类似实验积累了宝贵经验.

\end{document}