\documentclass[12pt, a4paper]{ctexart}
\usepackage[margin=2cm]{geometry}
\usepackage{libertine}

\usepackage{titlesec}
\usepackage{zhnumber}
\titleformat*{\section}{\Large\bfseries\raggedright}
\renewcommand{\thesection}{\zhnum{section}}
\renewcommand{\thesubsection}{\arabic{subsection}}

\setcounter{secnumdepth}{4}
\renewcommand{\theparagraph}{\thesubsubsection.\arabic{paragraph}}
\renewcommand{\paragraph}[1]{
    \refstepcounter{paragraph}
    {\bfseries\theparagraph\quad#1}\par
    \vspace{2pt}
    \noindent
}

\usepackage{enumitem}
\setlist[enumerate]{itemsep=2pt, parsep=0pt, partopsep=0pt, topsep=2pt}
\setlist[itemize]{itemsep=2pt, parsep=0pt, partopsep=0pt, topsep=2pt}
\linespread{1.2}

\usepackage{graphicx}

\usepackage{minted}
\setminted{
    breaklines=true, escapeinside=||, fontsize=\small, frame=lines,
    mathescape=$$, numbers=none, style=bw, tabsize=4
}

\usepackage{siunitx}

\begin{document}
    
\pagestyle{plain}
\thispagestyle{empty}

\noindent
\begin{tabular*}{\textwidth}{l @{\extracolsep{\fill}} r @{\extracolsep{6pt}} l}
    \LARGE{\textbf{实验报告}} & 计算机网络 & \textit{Computer Networking} \\
\end{tabular*}\\\\
\begin{tabular*}{\textwidth}{l l}
    \textbf{报告标题: } & \textbf{路由器及其基本配置} \\
    \textbf{学号: } & 19240212 \\
    \textbf{姓名: } & 华博文 \\
    \textbf{日期: } & 2025 年 11 月 3 日 \\
\end{tabular*}\\
\rule[2ex]{\textwidth}{2pt}

\section{实验目的}

本实验旨在系统掌握路由器接口的基本配置方法,包括接口 IP 地址的规划与设置、接口的启用与状态管理,这些技能是构建企业分段网络、实现部门间网络隔离与通信的基础;同时深入理解静态路由的工作原理,熟练掌握静态路由条目的配置命令,静态路由在拓扑稳定的小型网络、对路由安全性要求高的场景(如金融机构的核心网段互联)中具有不可替代的作用;最终学会通过 Ping 命令从端到端验证网络连通性,从而透彻理解计算机网络中不同网段间路由转发的机制,为复杂网络的多区域互联、路由策略规划等高级配置奠定扎实的技术基础。

\section{实验内容简要描述}

\begin{enumerate}
    \item 路由器基本配置:
    
    \begin{enumerate}
        \item 为 PC0、PC1、PC2、PC3 配置静态 IP 地址,基于网络规模和管理需求,将设备划分为 \mintinline{text}`192.168.100.0/24` 和 \mintinline{text}`192.168.200.0/24` 两个网段(PC0、PC1 属于前者,PC2、PC3 属于后者),通过网段划分实现初步的网络隔离与资源分区;
        \item 对路由器的 GigabitEthernet0/0、GigabitEthernet0/1 等接口进行 IP 地址配置,使其成为对应网段的网关,并启用与交换机连接的接口,激活物理层与数据链路层的通信能力,最终实现同网段内设备的直接通信。
    \end{enumerate}
    
    \item 配置静态路由:
    
    \begin{enumerate}
        \item 为路由器安装 HWIC-2T 模块(该模块提供 2 个串行接口,支持点到点广域网连接),模拟企业分支机构与总部的专线互联场景;
        \item 分别对两台路由器配置接口 IP、串口的 HDLC 封装协议(HDLC 是面向比特的同步链路层协议,具备高效帧传输能力)及静态路由条目,明确不同网段间的转发路径;
        \item 通过 Ping 命令从终端设备发起跨网段通信请求,验证静态路由配置的有效性,理解路由转发在网络层的实际运作流程。
    \end{enumerate}
\end{enumerate}

\section{实验步骤与结果分析}

\subsection{路由器基本配置}

\begin{figure}[H]
    \centering
    \includegraphics[width=\textwidth]{./img/screenshot_0000.png}
\end{figure}

\subsubsection{PC 地址设置}

分别为 PC0、PC1、PC2、PC3 配置静态 IP 地址:以 PC0 为例,在其网络配置界面选择 “Static” 模式,手动设置 IPv4 地址为 \mintinline{text}`192.168.100.11`,子网掩码 \mintinline{text}`255.255.255.0`(该掩码支持 254 台设备,满足实验网络规模需求);PC2、PC3 所在 \mintinline{text}`192.168.200.0/24` 网段,可配置 PC2 为 \mintinline{text}`192.168.200.11`、PC3 为 \mintinline{text}`192.168.200.12`,通过不同网段的 IP 分配为跨网段路由通信创造条件。

\begin{figure}[H]
    \centering
    \includegraphics[width=\textwidth]{./img/screenshot_0001.png}
\end{figure}

\subsubsection{路由器接口配置}

进入路由器全局配置模式后,进入接口配置视图,为该接口配置 IP 地址与子网掩码,使其成为 \mintinline{text}`192.168.100.0/24` 网段的网关;同理,对 GigabitEthernet0/1 接口配置 \mintinline{text}`192.168.200.1 255.255.255.0`,成为另一网段的网关。这些命令的本质是在网络层为接口赋予逻辑地址,使路由器能识别并转发不同网段的数据包。

\begin{figure}[H]
    \centering
    \includegraphics[width=\textwidth]{./img/screenshot_0002.png}
\end{figure}

\subsubsection{启用路由器接口}

在路由器接口配置界面中,勾选与交换机连接的 GigabitEthernet0/0 和 GigabitEthernet0/1 接口的 “Port Status” 启用选项,使接口状态从 “down” 变为 “up”。接口 “up/up” 状态表示物理层(如网线连接)和数据链路层(如以太网协议协商)均已就绪,路由器与交换机可正常收发数据帧,为同网段 PC 通信打通链路层通道。

\begin{figure}[H]
    \centering
    \includegraphics[width=\textwidth]{./img/screenshot_0003.png}
\end{figure}

\subsection{配置静态路由}

\begin{figure}[H]
    \centering
    \includegraphics[width=\textwidth]{./img/screenshot_0004.png}
\end{figure}

\subsubsection{安装 HWIC-2T 模块}

在路由器模块插槽中安装 HWIC-2T 模块,该模块提供的串行接口支持通过 V.35 线缆建立点到点广域网连接,为路由器间的静态路由配置提供物理基础。

\subsubsection{配置 Router0 和 Router1}

进入路由器特权模式和全局配置模式,依次配置接口 IP、串口封装协议及静态路由条目。执行 \mintinline{text}`show ip route` 命令后,路由表中会显示静态路由条目、目标网段和下一跳。

\begin{figure}[H]
    \centering
    \includegraphics[width=\textwidth]{./img/screenshot_0005.png}
\end{figure}

\begin{figure}[H]
    \centering
    \includegraphics[width=\textwidth]{./img/screenshot_0006.png}
\end{figure}

\subsubsection{连通性测试}

在 PC0 命令行中执行 \mintinline{bash}`ping 192.168.200.11`(PC2 的 IP):若配置正确,PC0 的 ICMP 请求将经 Router0→串口链路→Router1 转发至 PC2,PC2 返回的响应会沿原路径回传,最终在 PC0 终端显示 “Reply from 192.168.200.11” 及往返时间(RTT)、丢包率等统计信息,直观验证跨网段通信的畅通性。

\begin{figure}[H]
    \centering
    \includegraphics[width=\textwidth]{./img/screenshot_0007.png}
\end{figure}

\section{实验中遇到的问题及体会}

在实验中,曾因路由器接口未启用导致同网段 PC 无法通信:配置完接口 IP 后,未勾选 “Port Status” 启用选项,接口处于 “down” 状态,此时 PC0 ping PC1 显示 “Request timed out”。通过 \mintinline{text}`show interface GigabitEthernet0/0` 命令查看接口状态(“Line protocol” 和 “Physical” 均为 “down”),定位到问题后启用接口,通信恢复正常。这让我意识到:接口 “up” 状态是通信的基石,配置 IP 仅完成网络层逻辑设置,必须确保物理层和数据链路层同时就绪。

在静态路由配置中,曾因下一跳地址错误(将 Router0 的下一跳误写为自身串口 IP \mintinline{text}`10.0.0.1`)导致跨网段通信失败。通过对比两台路由器的串口 IP(Router0 为 \mintinline{text}`10.0.0.1`,Router1 为 \mintinline{text}`10.0.0.2`),明确 “下一跳需指向对端路由器的直连接口 IP” 的原则,修正后静态路由生效。这让我深刻理解:静态路由的 “下一跳” 是目标网段的 “下一个落脚点”,必须是相邻路由器的可达接口,否则数据包会因无转发路径而丢弃。

通过本次实验,我对路由器和静态路由的认知从理论走向实践:路由器通过接口 IP 划分网段,实现网络分段管理;静态路由是管理员手动建立的 “转发契约”,在拓扑稳定的场景(如小型企业互联)中可靠且易维护,但无法自动适应拓扑变化。Ping 命令则是网络诊断的 “利器”,从同网段到跨网段的连通性测试,将抽象的路由转发具象为终端响应,帮助我快速定位 IP、接口、路由等环节的问题。

总而言之,本次实验不仅让我掌握了 Cisco 设备配置、静态路由配置的实操技能,更构建了从网段划分到路由转发的完整网络层认知体系,为后续学习动态路由协议(如 OSPF、RIP)和复杂路由策略设计奠定了坚实基础。

\end{document}