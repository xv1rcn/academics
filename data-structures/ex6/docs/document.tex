\documentclass[12pt, a4paper]{ctexart}
\usepackage[margin=2cm]{geometry}
\usepackage{graphicx}
\usepackage{libertine}
\usepackage{siunitx}

\usepackage{titlesec}
\usepackage{zhnumber}
\titleformat*{\section}{\Large\bfseries\raggedright}
\renewcommand{\thesection}{\zhnum{section}}
\renewcommand{\thesubsection}{\arabic{subsection}}

\setcounter{secnumdepth}{4}
\renewcommand{\theparagraph}{\thesubsubsection.\arabic{paragraph}}
\renewcommand{\paragraph}[1]{
    \refstepcounter{paragraph}
    {\bfseries\theparagraph\quad#1}\par
    \vspace{2pt}
    \noindent
}

\usepackage{enumitem}
\setlist[enumerate]{itemsep=2pt, parsep=0pt, partopsep=0pt, topsep=2pt}
\setlist[itemize]{itemsep=2pt, parsep=0pt, partopsep=0pt, topsep=2pt}
\linespread{1.2}

\usepackage{minted}
\setminted{
    breaklines=true, escapeinside=||, fontsize=\small, frame=lines,
    mathescape=$$, numbers=none, style=bw, tabsize=4
}

\begin{document}
    
\pagestyle{plain}
\thispagestyle{empty}

\noindent
\begin{tabular*}{\textwidth}{l @{\extracolsep{\fill}} r @{\extracolsep{6pt}} l}
    \LARGE{\textbf{实验报告}} & 数据结构 & \textit{Data Structures} \\
\end{tabular*}\\\\
\begin{tabular*}{\textwidth}{l l}
    \textbf{报告标题: } & \textbf{大规模矩阵计算的性能测试} \\
    \textbf{学号: } & 19240212 \\
    \textbf{姓名: } & 华博文 \\
    \textbf{日期: } & 2025 年 11 月 18 日 \\
\end{tabular*}\\
\rule[2ex]{\textwidth}{2pt}

\section{实验环境}

\subsection{操作系统}

Ubuntu 24.04.3 LTS x86\_64,Linux 6.14.0-35-generic,AMD Ryzen 7 7700 (16) @ 5.3GHz,DDR5。

\subsection{编程工具}

NeoVim v0.12.0-dev,gcc 13.3.0,Python 3.12.3 (matplotlib == 3.6.3)。

\subsection{其他工具}

\LaTeX{} (\TeX\textit{studio})。

\section{实验内容及其完成情况}

\subsection{实验目的}

\begin{enumerate}
    \item 编写双精度矩阵乘法基准程序,理解计算密集型任务的性能特征;
    \item 测试并记录个人计算机能够处理的最大矩阵规模,分析时间和内存瓶颈;
    \item 探讨不同操作系统/配置对单进程内存使用的限制与可见性(RSS vs USS)。
\end{enumerate}

\subsection{实现方法}

本实验采用以下实现与测量策略:

\begin{itemize}
    \item \textbf{矩阵存储}:行主序,使用单个连续的 \mintinline{c++}`std::vector<double>` 存放元素,按索引计算偏移;
    
    \item \textbf{矩阵类型}:\mintinline{c++}`Matrix` 类,构造函数 \mintinline{c++}`Matrix(size_t rows, size_t cols)`,提供 \mintinline{c++}`operator()(i, j)` 和 \mintinline{c++}`operator*` 进行矩阵乘法;
    
    \item \textbf{乘法算法}:朴素三重循环实现,返回新的 \mintinline{c++}`Matrix` 对象;
    
    \item \textbf{性能测量}:使用 C++ 高精度时钟记录每次乘法所耗时间(秒),每个 \mintinline{c++}`N` 重复至少 $3$ 次取平均;
    
    \item \textbf{内存测量}:使用 POSIX \mintinline{c++}`getrusage(RUSAGE_SELF)` 获取进程峰值 RSS(Resident Set Size,单位 KB,报告中转换为 MB)。
\end{itemize}

\subsection{项目结构}

代码在本实验目录下:
\begin{itemize}
    \item \mintinline{text}`src/matrix.h`:\mintinline{c++}`Matrix` 类声明;
    \item \mintinline{text}`src/matrix.cpp`:\mintinline{c++}`Matrix` 类实现(\mintinline{c++}`operator*`、\mintinline{c++}`operator()` 等);
    \item \mintinline{text}`src/main.cpp`:基准运行器,接受参数 \mintinline{text}`START END STEP REPEATS OUTPUT_CSV`,生成 CSV;
    \item \mintinline{text}`src/plot.py`:读取 CSV,生成图像 \mintinline{text}`time_vs_n.png`、\mintinline{text}`mem_vs_n.png`、\mintinline{text}`comparison.png`。
\end{itemize}

\subsection{运行结果}

使用 \mintinline{bash}`g++ -O0 -std=c++20 main.cpp matrix.cpp -o matrix_bench` 编译项目,随后运行 \mintinline{bash}`./matrix_bench 100 2400 100 3 results.csv`,在 \mintinline{text}`results.csv` 中得到如下数据:

\begin{table}[h]
    \centering
    \begin{tabular}{l r r}
        \hline
        \mintinline{c++}`N` & \mintinline{c++}`avg_time_s` & \mintinline{c++}`peak_rss_MB` \\
        \hline
        100 & 0.00 & 7.63 \\
        200 & 0.01 & 7.63 \\
        300 & 0.03 & 7.63 \\
        400 & 0.08 & 8.89 \\
        500 & 0.15 & 11.64 \\
        600 & 0.26 & 15.02 \\
        700 & 0.42 & 18.82 \\
        800 & 0.61 & 23.25 \\
        900 & 0.87 & 28.35 \\
        1000 & 1.18 & 34.23 \\
        1100 & 1.57 & 40.71 \\
        1200 & 2.03 & 47.65 \\
        1300 & 2.58 & 55.23 \\
        1400 & 3.21 & 63.50 \\
        1500 & 3.94 & 72.44 \\
        1600 & 4.79 & 81.92 \\
        1700 & 5.75 & 91.89 \\
        1800 & 6.84 & 102.48 \\
        1900 & 8.07 & 113.84 \\
        2000 & 9.47 & 125.75 \\
        2100 & 11.07 & 138.40 \\
        2200 & 12.95 & 151.32 \\
        2300 & 14.87 & 165.04 \\
        2400 & 17.05 & 179.50 \\
        \hline
    \end{tabular}
\end{table}

由 \mintinline{bash}`python3 plot.py results.csv` 生成的图像:

\begin{figure}[H]
    \centering
    \includegraphics[width=\textwidth]{./img/time_vs_n.png}
\end{figure}

\begin{figure}[H]
    \centering
    \includegraphics[width=\textwidth]{./img/mem_vs_n.png}
\end{figure}

\begin{figure}[H]
    \centering
    \includegraphics[width=\textwidth]{./img/comparison.png}
\end{figure}

\section{数据分析与讨论}

\subsection{性能标度}

理论上,朴素矩阵乘法的时间复杂度为 $\mathcal{O}\left(N^3\right)$;从实验数据可通过对数拟合检验时间随 $N^3$ 增长的趋势。对于小规模(L1 / L2 缓存可容纳)矩阵,缓存命中使实际常数显著更小;当问题规模超过缓存并进入内存访问瓶颈时,时间增长速率加剧。

\subsection{内存限制与崩溃点}

峰值 RSS 反映的是进程在系统可见的物理内存占用(包含共享库占用);如果进程分配失败会抛出 \mintinline{c++}`std::bad_alloc`,程序会在 CSV 中记录崩溃前的最大 N。

\subsection{潜在优化}

\begin{itemize}
    \item 使用块矩阵乘法(blocking / tiling)减少缓存未命中;
    \item 利用多线程(OpenMP)或调用高性能 BLAS(例如 OpenBLAS / Intel MKL);
    \item 避免在基准中频繁分配临时矩阵,可使用预分配输出矩阵并实现 \mintinline{c++}`multiply_to` 接口以排除分配开销对时间测量的影响。
\end{itemize}

\subsection{结论}

本实验提供了一个简单、可重复的矩阵乘法基准框架,满足作业要求:使用双精度、记录时间与峰值内存、给出 CSV 原始数据并绘制对比图表。基准结果显示时间复杂度接近 $\mathcal{O}\left(N^3\right)$,内存消耗按 $\mathcal{O}\left(N^2\right)$ 增长。要进一步提高性能或扩展测试,可采用阻塞、并行或调用 BLAS 实现。

\appendix

\section{完整代码}


\subsection{\mintinline{text}`matrix.h`}

\inputminted{c++}{../src/matrix.h}

\subsection{\mintinline{text}`matrix.cpp`}

\inputminted{c++}{../src/matrix.cpp}

\subsection{\mintinline{text}`main.cpp`}

\inputminted{c++}{../src/main.cpp}

\subsection{\mintinline{text}`plot.py`}

\inputminted{python}{../src/plot.py}

\end{document}