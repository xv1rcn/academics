\documentclass[12pt, a4paper]{ctexart}
\usepackage[margin=2cm]{geometry}
\usepackage{graphicx}
\usepackage{libertine}
\usepackage{siunitx}

\usepackage{titlesec}
\usepackage{zhnumber}
\titleformat*{\section}{\Large\bfseries\raggedright}
\renewcommand{\thesection}{\zhnum{section}}
\renewcommand{\thesubsection}{\arabic{subsection}}

\setcounter{secnumdepth}{4}
\renewcommand{\theparagraph}{\thesubsubsection.\arabic{paragraph}}
\renewcommand{\paragraph}[1]{
    \refstepcounter{paragraph}
    {\bfseries\theparagraph\quad#1}\par
    \vspace{2pt}
    \noindent
}

\usepackage{enumitem}
\setlist[enumerate]{itemsep=2pt, parsep=0pt, partopsep=0pt, topsep=2pt}
\setlist[itemize]{itemsep=2pt, parsep=0pt, partopsep=0pt, topsep=2pt}
\linespread{1.2}

\usepackage{minted}
\setminted{
    breaklines=true, escapeinside=||, fontsize=\small, frame=lines,
    mathescape=$$, numbers=none, style=bw, tabsize=4
}

\begin{document}
    
\pagestyle{plain}
\thispagestyle{empty}

\noindent
\begin{tabular*}{\textwidth}{l @{\extracolsep{\fill}} r @{\extracolsep{6pt}} l}
    \LARGE{\textbf{实验报告}} & 数据结构 & \textit{Data Structures} \\
\end{tabular*}\\\\
\begin{tabular*}{\textwidth}{l l}
    \textbf{报告标题: } & \textbf{线性表归并} \\
    \textbf{学号: } & 19240212 \\
    \textbf{姓名: } & 华博文 \\
    \textbf{日期: } & 2025 年 9 月 23 日 \\
\end{tabular*}\\
\rule[2ex]{\textwidth}{2pt}

\section{实验环境}

\subsection{操作系统}

Ubuntu 24.04.3 LTS x86\_64, Linux 6.14.0-28-generic. AMD Ryzen 7 7700 (16) @ 5.3GHz.

\subsection{编程工具}

NeoVim v0.11.1, gcc 13.3.0, GNU Make 4.3, cmake 3.28.3, Python 3.12.3 (matplotlib == 3.6.3).

\subsection{其他工具}

\LaTeX{} (\TeX\textit{studio}).

\section{实验内容及其完成情况}

\subsection{实验目的}

本次实验旨在通过亲手实现线性表的归并功能, 深入理解顺序表与链表两种基本存储结构的特性, 掌握归并算法的核心思想, 并强化编程实现中的指针操作与边界条件处理能力. 具体目标包括:

\begin{enumerate}
    \item 明确存储结构差异: 通过实现两种线性表的归并操作, 体会顺序表 (连续内存) 与链表 (指针链接) 在存储方式上的本质区别, 理解两者在归并思路与代码细节上的差异;
    \item 掌握归并算法核心: 深入理解归并算法 ``比较——插入'' 的逻辑, 能够用代码精确实现有序线性表的合并, 确保结果仍保持有序;
    \item 提升编程健壮性: 重点训练指针操作 (尤其是链表中的指针移动、判空与连接), 并处理边界情况 (如空表、重复元素、单表元素全大于另一表等), 编写高健壮性代码.
\end{enumerate}

\subsection{实验内容}

本次实验的核心任务是实现有序线性表 (从小到大排列) 的归并功能, 分为两个子任务, 且每项任务需提供至少两种解决方案.

\subsubsection{核心任务}

\begin{itemize}
    \item 任务 1: 有序顺序表归并——将两个有序顺序表合并为一个新的有序顺序表;
    \item 任务 2: 有序链表归并——将两个有序链表合并为一个新的有序链表.
\end{itemize}

\subsubsection{实验要求}

\begin{enumerate}
    \item 健壮性: 需覆盖空表、重复元素、极端值 (如单表元素全大/全小) 等边界场景;
    \item 语言: 采用 C++ 语言实现, 可借助 STL 容器但需明确自定义与 STL 实现的差异;
    \item 性能分析: 从时间、空间复杂度角度对比不同方案, 给出量化结论;
    \item 拓展测试: 通过大规模数据 (如数据规模 $n$ 从 $0$ 到 $100000$) 测试, 以图表形式展示各方案性能优劣.
\end{enumerate}

\subsection{实验过程}

\subsubsection{整体实现框架}

本次实验设计四种容器类封装线性表操作, 分别为:

\begin{itemize}
    \item \mintinline{cpp}{ordered_array<T>}: 自定义动态数组实现的有序顺序表;
    \item \mintinline{cpp}{ordered_array_stl<T>}: 基于 STL \mintinline{cpp}{std::vector} 的有序顺序表;
    \item \mintinline{cpp}{ordered_list<T>}: 自定义单向链表实现的有序链表;
    \item \mintinline{cpp}{ordered_list_stl<T>}: 基于 STL \mintinline{cpp}{std::list} 的有序链表.
\end{itemize}

实验流程通过两个核心函数驱动:

\begin{itemize}
    \item \mintinline{cpp}{test<C>(std::ostream &out)}: 验证容器功能正确性, 覆盖基础与边界用例;
    \item \mintinline{cpp}{profile<C, T>(size_t n, std::ostream &out)}: 在指定数据规模 $n$ 下测试性能, 结果输出至控制台与 \mintinline{text}{profile.txt}.
\end{itemize}

主函数 (\mintinline{text}{main.cpp}) 代码如下, 负责调用测试与性能分析:

\begin{minted}{C++}
int main() {
    test<ordered_array<int>>();
    test<ordered_array_stl<int>>();
    test<ordered_list<int>>();
    test<ordered_list_stl<int>>();
    dout << "All container tests finished!" << std::endl << std::endl;
    
    std::ofstream fout("profile.txt");
    cf_ostream dout(std::cout, fout);
    
    for (size_t n = 0; n <= 100'000; n += 1'000) {
        dout << "Profiling with n = " << n << std::endl;
        profile<ordered_array<int>, int>(n, dout);
        profile<ordered_array_stl<int>, int>(n, dout);
        profile<ordered_list<int>, int>(n, dout);
        profile<ordered_list_stl<int>, int>(n, dout);
        dout << "Finished profiling for n = " << n << std::endl << std::endl;
    }
    dout << "All container profiles finished!" << std::endl << std::endl;
    
    return 0;
}
\end{minted}

\subsubsection{功能测试实现}

为验证容器正确性, \mintinline{text}{test.cpp} 设计多组用例, 覆盖 ``基础功能 + 边界场景'', 确保代码在极端情况下仍能正常运行. 测试逻辑通过 \mintinline{cpp}{assert} 断言验证, 若某用例失败则直接终止程序, 便于定位问题.

以归并功能测试为例, 代码验证合并后有序性, 同时测试空表、重复元素等场景:

\begin{minted}{C++}
template <typename C>
void test(std::ostream &out = std::cout) {
    out << "Testing " << typeid(C).name() << std::endl;
    
    ...
    
    // 合并后升序
    C b;
    for (int i = 0; i < 5; ++i) b.ordered_insert(i * 2);
    C c;
    for (int i = 0; i < 5; ++i) c.ordered_insert(i * 2 + 1);
    b.merge(c);
    for (size_t i = 1; i < b.size(); ++i) assert(b[i - 1] <= b[i]);
    
    ...
    
    // 自我赋值 / 合并
    d = d;
    for (size_t i = 0; i < d.size(); ++i) assert(d[i] == i);
    e.merge(e);
    for (size_t i = 1; i < e.size(); ++i) assert(e[i - 1] <= e[i]);
    
    // 多次 clear / resize / merge
    a.clear();
    a.resize(3);
    a.clear();
    a.push_back(1);
    a.push_back(2);
    a.push_back(3);
    d = a;
    d.merge(a);
    d.clear();
    d.resize(2);
    d.push_back(99);
    assert(d.size() == 3 && d[2] == 99);
    
    // 边界插入 / 删除
    a.clear();
    a.push_back(10);
    a.insert(0, 5);
    a.insert(a.size(), 20);
    assert(a[0] == 5 && a[a.size() - 1] == 20);
    a.erase(0);
    a.erase(a.size() - 1);
    assert(a[0] == 10);
    
    ...
    
    out << "All tests passed for " << typeid(C).name() << std::endl;
}
\end{minted}

所有测试用例通过后, 控制台输出 \mintinline{text}{All tests passed for [容器名]}, 例如 \mintinline{text}{All tests passed for ordered_array<int>}, 标志容器功能正确.

\subsubsection{顺序表归并的两种实现方案}

顺序表的核心特性是 ``连续内存 + 随机访问'', 即通过下标可直接定位任意元素, 这一特性直接影响归并操作的实现逻辑与性能.
,
\paragraph{自定义顺序表 (\mintinline{cpp}{ordered_array<T>})}

\textbf{存储结构}: 基于动态数组 (\mintinline{cpp}{T *data}) 实现, 内存连续分配, 通过 \mintinline{cpp}{size} 记录当前元素个数、\mintinline{cpp}{capacity} 记录数组最大容量, 扩容时按 2 倍比例分配新内存 (减少扩容次数).

\textbf{归并核心逻辑}:

\begin{enumerate}
    \item 预分配内存: 创建新动态数组 \mintinline{cpp}{new_data}, 大小为两表长度之和 (\mintinline{cpp}{_size + other._size}), 避免归并过程中频繁扩容;
    \item 双指针遍历: 用 \mintinline{cpp}{T *pt1 = data} (指向当前表)、\mintinline{cpp}{T *pt2 = other.data} (指向待合并表) 遍历两表, 比较 \mintinline{cpp}{data[pt1]} 与 \mintinline{cpp}{other.data[pt2]} 的大小;
    \item 元素复制: 将较小值存入 \mintinline{cpp}{*pt} (初始 \mintinline{cpp}{T *pt = new_data}), 并移动对应指针 (\mintinline{cpp}{pt1++} 或 \mintinline{cpp}{pt2++});,
    \item 剩余元素追加: 当某一表遍历完毕 (如 \mintinline{cpp}{pt1 >= data + _size}), 将另一表剩余元素复制到 \mintinline{cpp}{new_data};
    \item 更新当前表: 释放原数组内存, 将 \mintinline{cpp}{data} 指向 \mintinline{cpp}{new_data}, 并更新 \mintinline{cpp}{size} 为 \mintinline{cpp}{_size + other._size}.
\end{enumerate}

归并操作核心代码如下:

\begin{minted}{C++}
template <typename T>
void ordered_array<T>::merge(const ordered_array<T> &other) {
    if (!other._size) return;
    T *new_data = new T[_size + other._size]{};
    T *pt1 = data, *pt2 = other.data, *pt = new_data;
    while (pt1 < data + _size && pt2 < other.data + other._size) {
        if (*pt1 < *pt2)
        *pt++ = *pt1++;
        else
        *pt++ = *pt2++;
    }
    while (pt1 < data + _size) *pt++ = *pt1++;
    while (pt2 < other.data + other._size) *pt++ = *pt2++;
    delete[] data;
    data = new_data;
    _size += other._size;
    _capacity = _size;
}
\end{minted}

\textbf{复杂度分析}: 

\begin{itemize}
    \item 时间复杂度: 遍历两表共 $m+n$ 次, 元素复制操作同样为 $m+n$ 次, 无嵌套循环, 故为 $\mathcal{O}(m+n)$;
    \item 空间复杂度: 需额外分配 $m+n$ 大小的动态数组, 故为 $\mathcal{O}(m+n)$ (不含输入数据本身的空间).
\end{itemize}

\textbf{性能数据}: 从 \mintinline{text}{profile.txt} 提取数据, 该容器归并耗时是四种容器中最快的——因无 STL 通用逻辑的额外开销, 且自定义内存管理更紧凑.

\paragraph{基于 STL 的顺序表 (\mintinline{cpp}{ordered_array_stl<T>})}

\textbf{存储结构}: 底层使用 STL 容器 \mintinline{cpp}{std::vector}, 无需手动管理内存 (STL 自动扩容、释放), 通过迭代器 (\mintinline{cpp}{std::vector<T>::iterator}) 实现元素访问.

\textbf{归并核心逻辑}: 借助 STL 算法 \mintinline{cpp}{std::merge}, 该算法已高度优化, 支持任意可迭代容器的归并. 实现步骤如下:

\begin{enumerate}
    \item 创建临时容器: 定义 \mintinline{cpp}{std::vector<T> result}, 用于存储归并结果;
    \item 调用STL算法: 通过 \mintinline{cpp}{std::merge(data.begin(), data.end(), other.data.begin(), other.data.end(), std::back_inserter(result))} 执行归并:
    
    \begin{itemize}
        \item \mintinline{cpp}{data.begin()} / \mintinline{cpp}{data.end()}: 当前表的首尾迭代器;
        \item \mintinline{cpp}{std::back_inserter(temp)}: 自动调用 \mintinline{cpp}{result.push_back()}, 避免手动管理 \mintinline{cpp}{temp} 的大小;
    \end{itemize}
    
    \item 更新当前表: 将 \mintinline{cpp}{data} 替换为 \mintinline{cpp}{result} (利用 \mintinline{cpp}{std::vector} 的移动语义, 减少复制开销).
\end{enumerate}

归并操作核心代码如下:

\begin{minted}{C++}
template <typename T>
void ordered_array_stl<T>::merge(const ordered_array_stl<T> &other) {
    std::vector<T> result;
    result.reserve(data.size() + other.data.size());
    std::merge(data.begin(), data.end(), other.data.begin(), other.data.end(), std::back_inserter(result));
    data = std::move(result);
}
\end{minted}

\textbf{复杂度分析}:

\begin{itemize}
    \item 时间复杂度: 与自定义实现一致, 为 $\mathcal{O}(m+n)$——\mintinline{cpp}{std::merge} 本质仍是双指针遍历, 仅迭代器访问的常数因子略高;
    \item 空间复杂度: \mintinline{cpp}{result} 需存储 $m+n$ 个元素, 故为 $\mathcal{O}(m+n)$.
\end{itemize}

\textbf{性能数据}: 归并耗时略高于自定义顺序表——因 STL 算法需兼容所有可迭代容器 (如 \mintinline{cpp}{std::list}、\mintinline{cpp}{std::deque}), 通用逻辑带来轻微开销, 但开发效率显著提升.

\subsubsection{链表归并的两种实现方案}

链表的核心特性是 ``离散内存 + 指针链接'', 即元素存储在不连续的内存块 (节点) 中, 通过指针 (\mintinline{cpp}{next}) 连接, 不支持随机访问, 需通过指针移动遍历.

\paragraph{自定义链表 (\mintinline{cpp}{ordered_list<T>})}

\textbf{存储结构}: 单向链表, 节点定义如下:

\begin{minted}{C++}
template <typename T>
struct ordered_list<T>::Node {
    T value;
    Node *next;
    Node(const T &val, Node *nxt = nullptr) : value(val), next(nxt) {}
};
\end{minted}

容器通过 \mintinline{cpp}{Node *head} 指向表头, \mintinline{cpp}{size_t size} 记录节点个数, 空表时 \mintinline{cpp}{head == nullptr}.

\textbf{归并核心逻辑}:

\begin{enumerate}
    \item 哨兵节点优化: 创建哨兵节点 \mintinline{cpp}{Node dummy(T{})}, 避免处理 ``表头为空'' 的边界条件 (如两表均为空时, 哨兵的 \mintinline{cpp}{next} 直接为 \mintinline{cpp}{nullptr});
    \item 双指针遍历: 用 \mintinline{cpp}{mtail = &dummy} 指向新链表尾部, \mintinline{cpp}{l1 = head} (当前表头)、\mintinline{cpp}{l2 = other.head} (待合并表头) 分别遍历两表;
    \item 节点创建与链接: 比较\mintinline{cpp}{l1->value} 与 \mintinline{cpp}{l2->value}, 通过 \mintinline{cpp}{new Node(|值|)} 创建新节点并链接到 \mintinline{cpp}{mtail->next}, 同时移动 \mintinline{cpp}{mtail} 与对应指针 (\mintinline{cpp}{l1} 或 \mintinline{cpp}{l2});
    \item 剩余元素处理: 当某一表遍历完毕 (\mintinline{cpp}{l1 == nullptr} 或 \mintinline{cpp}{l2 == nullptr}), 循环创建剩余元素的新节点并链接;
    \item 当前表更新: 调用 \mintinline{cpp}{clear()} 释放原有节点, 将 \mintinline{cpp}{head} 指向哨兵的下一个节点 (新表头), 定位新表尾并更新长度为两表之和.
\end{enumerate}

归并操作核心代码如下:

\begin{minted}{C++}
template <typename T>
void ordered_list<T>::merge(const ordered_list<T> &other) {
    size_t old_size = _size;
    Node dummy(T{});
    Node *mtail = &dummy;
    Node *l1 = head, *l2 = other.head;
    while (l1 && l2) {
        if (l1->value < l2->value) {
            mtail->next = new Node(l1->value);
            l1 = l1->next;
        } else {
            mtail->next = new Node(l2->value);
            l2 = l2->next;
        }
        mtail = mtail->next;
    }
    while (l1) {
        mtail->next = new Node(l1->value);
        mtail = mtail->next;
        l1 = l1->next;
    }
    while (l2) {size_t old_size = _size;
        Node dummy(T{});
        Node *mtail = &dummy;
        Node *l1 = head, *l2 = other.head;
        while (l1 && l2) {
            if (l1->value < l2->value) {
                mtail->next = new Node(l1->value);
                l1 = l1->next;
            } else {
                mtail->next = new Node(l2->value);
                l2 = l2->next;
            }
            mtail = mtail->next;
        }
        while (l1) {
            mtail->next = new Node(l1->value);
            mtail = mtail->next;
            l1 = l1->next;
        }
        mtail->next = new Node(l2->value);
        mtail = mtail->next;
        l2 = l2->next;
    }
    clear();
    head = dummy.next;
    tail = head;
    if (tail) while (tail->next) tail = tail->next;
    _size = old_size + other._size;
}
\end{minted}

\textbf{复杂度分析}:

\begin{itemize}
    \item 时间复杂度: 遍历两表共 $m+n$ 次, 指针调整操作为 $m+n$ 次, 故为 $\mathcal{O}(m+n)$;
    \item 空间复杂度: 仅需1个哨兵节点 (常数空间), 故为 $\mathcal{O}(1)$ (不含结果节点本身的空间——结果复用原节点内存).
\end{itemize}

\textbf{性能数据}: 归并耗时高于两种顺序表——因指针跳转的硬件开销 (CPU 缓存命中率低, 连续内存访问更高效).

\paragraph{基于 STL 的链表 (\mintinline{cpp}{ordered_list_stl<T>})}

\textbf{存储结构} 底层使用 STL 容器 \mintinline{cpp}{std::list}, 该容器为双向链表 (节点含 \mintinline{cpp}{prev} 与 \mintinline{cpp}{next} 指针), STL 自动管理节点内存与指针调整.

\textbf{归并核心逻辑}: 直接调用 \mintinline{cpp}{std::list} 的成员函数 \mintinline{cpp}{merge}, 该函数为链表优化设计 (需要复制元素, 否则 \mintinline{cpp}{other.lst} 将被置空). 双向链表的指针调整虽比单向链表复杂, 但 STL 内部通过汇编级优化减少开销.

归并操作核心代码如下:

\begin{minted}{C++}
template <typename T>
void ordered_list_stl<T>::merge(const ordered_list_stl<T> &other) {
    std::list<T> other_copy = other.data;
    data.merge(other_copy);
}
\end{minted}

\textbf{复杂度分析}:

\begin{itemize}
    \item 时间复杂度: 与自定义单向链表一致, 为 $\mathcal{O}(m+n)$——虽需同时调整 \mintinline{cpp}{prev} 与 \mintinline{cpp}{next} 指针, 但操作次数仍与 $m+n$ 成正比;
    \item 空间复杂度: 指针调整虽为 $\mathcal{O}(1)$, 但需要复制 \mintinline{cpp}{other}, 故为 $\mathcal{O}(n)$.
\end{itemize}

\textbf{性能数据}: 归并耗时是四种容器中最慢的——因双向链表需维护两个指针, 每次节点链接的操作数是单向链表的 2 倍, 常数因子最大.

\subsection{性能分析}

\subsubsection{时间性能与时间复杂度对比}

时间复杂度描述 ``算法耗时随数据规模增长的趋势'', 而实际性能受 ``常数因子'' (如内存访问方式、指针操作次数) 影响. 结合 \mintinline{text}{profile.txt} 中大规模测试数据 ($n$ 从 $1000$ 到 $100000$), 两种维度的对比如下:

\paragraph{核心操作时间复杂度汇总}
\begin{table}[H]
    \centering
    \label{tab:time_complexity}
    \begin{tabular}{|c|c|c|c|}
        \hline
        \textbf{操作类型} & \textbf{顺序表} & \textbf{链表} & \textbf{核心差异原因} \\
        \hline
        \mintinline{cpp}{merge} & $\mathcal{O}(m+n)$ & $\mathcal{O}(m+n)$ & / \\
        \hline
        \mintinline{cpp}{ordered_insert} & $\mathcal{O}(\log n)$ & $\mathcal{O}(n)$ & 顺序表二分查找, 线性链表遍历 \\
        \hline
        \mintinline{cpp}{contains} & $\mathcal{O}(\log n)$ & $\mathcal{O}(n)$ & 顺序表二分查找, 线性链表遍历 \\
        \hline
    \end{tabular}
\end{table}

\paragraph{实际性能差异(以 $n=90000$ 为例)}

从 \mintinline{text}{profile.txt} 提取关键数据, 各容器的核心操作耗时对比如下 (单位: 秒):

\begin{itemize}
    \item 归并操作: \mintinline{cpp}{ordered_array<int>} (\num{6.870e-4}) $<$ \mintinline{cpp}{ordered_array_stl<int>} (\num{3.475e-3}) $<$ \mintinline{cpp}{ordered_list<int>} (\num{6.084e-3}) $<$ \mintinline{cpp}{ordered_list_stl<int>} (\num{1.014e-2});
    \item 有序插入: \mintinline{cpp}{ordered_array_stl<int>} (\num{2.200e-1}) 远低于 \mintinline{cpp}{ordered_list<int>} (\num{1.547e1})——顺序表元素移动基于连续内存块复制, CPU 可批量处理, 而链表需逐个节点跳转;
    \item 查找操作: \mintinline{cpp}{ordered_array_stl<int>} (\num{1.538e-2}) 与 \mintinline{cpp}{ordered_list<int>} (\num{3.770e1}) 差距悬殊——$\mathcal{O}(\log n)$ 与 $\mathcal{O}(n)$ 在大规模数据下的差异呈指数级放大 (如 $n=10^5$ 时, $\log_2 n \approx 16.61$, 而 $n=10^5$).
\end{itemize}

\subsubsection{空间性能对比}

空间复杂度描述 ``算法额外占用内存随数据规模增长的趋势'', 两种线性表的差异如下:

\begin{itemize}
    \item 顺序表: 归并需分配 $m+n$ 大小的连续内存 (如自定义顺序表的 \mintinline{cpp}{new_data}、STL 顺序表的 \mintinline{cpp}{result}), 空间复杂度 $\mathcal{O}(m+n)$; 但连续内存访问符合 CPU 缓存的 ``局部性原理'' (访问一个元素时, 相邻元素会被预加载到缓存), 缓存命中率高, 间接降低时间开销.
    \item 链表: 归并仅需调整指针 (如自定义链表的哨兵节点、STL 链表的内部指针), 额外空间为常数 ($\mathcal{O}(1)$); 但节点内存离散分布, 每次指针跳转都会破坏缓存局部性, 缓存命中率低, 即使空间开销小, 时间开销仍高于顺序表.
\end{itemize}

当 $m=n=10^5$ 时, 顺序表需额外分配 \num{2e5} 个 \mintinline{cpp}{int} (约 $\qty{781}{kB}$, \mintinline{cpp}{int} 占 4 字节), 而链表仅需 1 个哨兵节点与 \num{1e5} 个 \mintinline{cpp}{int} (约 $\qty{391}{kB}$); 但顺序表归并耗时约 \num{e-3}\si{s}, 链表约 \num{e-2}\si{s}, 时间差距远大于空间优势.

\subsubsection{大规模数据测试图表}

通过 Python 的 \mintinline{python}{matplotlib} 库处理 \mintinline{text}{profile.txt} 数据, 绘制 ``归并耗时随数据规模 $n$ 的变化曲线''. 图表趋势表明:

\begin{enumerate}
    \item 所有方案的耗时随 $n$ 线性增长, 符合 $\mathcal{O}(n)$ 复杂度的理论预期;
    \item 四种容器的耗时曲线始终保持固定间距 (顺序表 < 自定义链表 < STL 链表), 说明常数因子的影响稳定;
    \item 当 $n$ 从 $1000$ 增长到 $100000$ 时, 四种容器的耗时差距扩大, 规模越大, 顺序表的优势越明显.
\end{enumerate}

\begin{figure}[H]
    \centering
    \includegraphics[width=\textwidth]{./img/profile_merge.png}
\end{figure}

\subsection{实验结论}

\begin{enumerate}
    \item \textbf{存储结构决定性能上限}:
    
    \begin{itemize}
        \item 顺序表 (尤其是 \mintinline{cpp}{ordered_array<int>}) 在大规模数据场景下表现最优, 其核心优势是 ``连续内存 + 随机访问''——不仅时间复杂度低 (如查找 $\mathcal{O}(\log n)$), 且 CPU 缓存利用率高, 常数因子小;
        \item 链表 (如 \mintinline{cpp}{ordered_list_stl<int>}) 仅在 ``内存碎片化严重'' 或 ``频繁插入删除 (表头 / 表尾)'' 场景下有优势, 大规模数据下因指针开销与线性查找, 性能显著落后.
    \end{itemize}
    
    \item \textbf{STL 实现的实用性权衡}:
    
    \begin{itemize}
        \item 基于 STL 的容器 (\mintinline{cpp}{ordered_array_stl<int>}、\mintinline{cpp}{ordered_list_stl<int>}) 代码量相比自定义实现大幅减少, 且内存管理、边界处理更健壮 (如自动扩容、避免内存泄漏);
        \item 虽 STL 实现存在轻微性能损耗 (如 \mintinline{cpp}{ordered_array_stl<int>} 比 \mintinline{cpp}{ordered_array<int>} 慢), 但在实际开发中, ``开发效率'' 与 ``代码稳定性'' 通常优先于 ``极致性能'', 故推荐优先使用 STL.
    \end{itemize}
    
    \item \textbf{复杂度与实际性能的关联}:
    
    \begin{itemize}
        \item 时间复杂度的 ``阶数'' (如 $\mathcal{O}(n)$、$\mathcal{O}(\log n)$) 决定了算法的 ``可扩展性'' (数据规模增长时的耗时增长趋势);
        \item 存储结构的特性 (连续 / 离散内存) 决定了 ``常数因子'' (单次操作的硬件耗时), 两者共同决定算法的实际性能——理论分析需结合硬件特性 (如 CPU 缓存), 才能准确评估算法优劣.
    \end{itemize}
\end{enumerate}

本次实验通过 ``理论推导 + 代码实现 + 性能测试'' 的闭环, 深化了对线性表存储结构与归并算法的理解, 为后续学习归并排序、外部排序等复杂算法奠定了基础.

\section{实验总结}

本次线性表归并实验以顺序表和链表的归并功能实现为核心, 通过自定义代码与 STL 封装两种方式, 完成了功能开发、边界场景测试、性能数据采集与时间复杂度验证. 实验过程中, 不仅解决了模板编程、内存管理、数据波动等常见问题, 也借助生成的性能图表加深了对线性表存储特性的理解; 同时, 针对部分技术细节的深层原理, 目前仍存在待进一步探索的空间. 以下从问题解决、待探索方向与图表应用三方面进行总结.

\subsection{实验中解决的关键问题与思路}

在实验编码与调试阶段, 遇到了若干影响功能实现与数据准确性的问题. 通过查阅资料、逐步排查与反复测试, 最终找到解决方案, 为实验的正常开展提供了保障.

\subsubsection{模板类名 ``长度 + 原名 + 签名'' 的解析疑惑}

实验初期调试模板类时, 通过 \mintinline{cpp}{typeid(T).name()} 查看类类型, 得到的结果并非预期的 \mintinline{cpp}{ordered_array<int>}, 而是 \mintinline{cpp}{13ordered_arrayIiE} 这类包含数字与特殊符号的字符串 (例如 \mintinline{cpp}{17ordered_array_stlIiE} 对应 \mintinline{cpp}{ordered_array_stl<int>}), 最初误以为是模板实例化出现错误. 经查阅 C++ 编译原理相关资料后了解到, 这是编译器的名字修饰 (\textit{Name Mangling}) 机制——由于 C++ 支持函数重载与模板实例化, 为区分不同参数的模板类 / 函数, 编译器会将类名长度、模板参数类型等信息编码为类名: 前缀数字代表类名字符数 (如 \mintinline{cpp}{13ordered_arrayIiE} 中 ``ordered\_array'' 共 13 个字符), 后缀符号代表模板参数类型 (如 \mintinline{cpp}{IiE} 对应 \mintinline{cpp}{int}, \mintinline{cpp}{I} 表示模板参数开始, \mintinline{cpp}{i} 表示 \mintinline{cpp}{int}, \mintinline{cpp}{E} 表示结束).

为避免后续分析混淆, 可以通过 gcc 的\mintinline{bash}{c++filt} 命令对修饰后的类名进行反解析 (如执行 \mintinline{bash}{c++filt 13ordered_arrayIiE} 可得到 \mintinline{cpp}{ordered_array<int>}), 同时在 \mintinline{text}{profile.txt} 日志中注明修饰后类名与原始类名的对应关系, 确保性能数据匹配准确. 由于目前仍能清晰分辨类, 本次实验中仅用 Python 正则表达式捕获后作简单解析.

\subsubsection{链表归并中的内存重复释放问题}

自定义链表 \mintinline{cpp}{ordered_list<T>} 的归并函数初期尝试 ``复用原节点'' (直接调整 \mintinline{cpp}{next} 指针链接原节点), 但运行时频繁触发 \mintinline{text}{double free} 错误. 经排查发现, \mintinline{cpp}{other} 对象析构时会释放已合并到当前表的节点, 导致同一节点被两次释放. 针对这一问题, 实验中测试了两种解决方案:

\begin{itemize}
    \item 一是 ``新建节点存储结果'': 通过 \mintinline{cpp}{new Node(l1->value)} 复制原节点数据到新节点, 合并后调用 \mintinline{cpp}{clear()} 释放当前表原节点, 彻底避免所有权冲突. 从归并耗时曲线来看, 该方案耗时比 ``复用节点'' 高约 \qtyrange[range-phrase={\,\textasciitilde\,}]{15}{20}{\%} (如 $n=100000$ 时, 新建节点耗时 \qty{0.0063}{s}, 复用节点耗时 \qty{0.0052}{s}), 但能确保内存安全, 适合对稳定性要求较高的场景. 
    \item 二是 ``转移节点所有权'': 通过 \mintinline{cpp}{const_cast<ordered_list<T>&>(other).head = nullptr} 将 \mintinline{cpp}{other} 的表头置空, 使其析构时不再释放节点. 该方案耗时更接近顺序表, 但需严格保证 \mintinline{cpp}{other} 对象后续不再被访问, 适合性能优先且能控制对象生命周期的场景.
\end{itemize}

\subsubsection{性能测试数据的波动与修正}

在大规模性能测试 ($n$ 从 $1000$ 到 $100000$) 中, 相同 $n$ 值下归并耗时存在 \qty{\pm8}{\%} 的波动 (如 $n=50000$ 时, \mintinline{cpp}{ordered_array_stl<int>} 的 \mintinline{cpp}{merge} 耗时在 \qtyrange[range-phrase={\,\textasciitilde\,}]{0.0032}{0.0035}{s} 之间), 导致初始绘制的曲线存在 ``毛刺'', 难以直观观察性能趋势. 分析认为, 波动源于系统后台进程 (如杀毒软件扫描、内存调度) 占用 CPU 资源, 影响测试线程的连续执行.

为改善数据准确性, 实验测试前关闭后台非必要进程, 减少外界干扰. 修正后, 曲线平滑度明显提升, 能清晰呈现 ``耗时随 $n$ 线性增长'' 的趋势, 与 $\mathcal{O}(m+n)$ 的理论复杂度基本吻合.


\subsection{待深入探索的问题}

实验虽完成了核心目标, 但针对部分技术细节的深层原理与优化方向, 目前仍存在未明确的内容, 需在后续学习中进一步研究.

\subsubsection{STL 链表 \mintinline{cpp}{merge} 的性能瓶颈}

从归并耗时曲线可见, 基于 \mintinline{cpp}{std::list} 的 \mintinline{cpp}{ordered_list_stl<int>}, 其归并耗时始终高于自定义单向链表 (如 $n=100000$ 时, 前者耗时 \qty{0.0102}{s}, 后者 \qty{0.0063}{s}). 初步推测与双向链表需维护 \mintinline{cpp}{prev} 指针有关, 但目前暂未明确 \mintinline{cpp}{prev} 指针调整的具体开销占比——一方面, 缺乏指令级性能分析工具 (如 \mintinline{bash}{perf}、\mintinline{bash}{VTune}), 无法量化 ``\mintinline{cpp}{prev} 指针赋值'' 与 ``\mintinline{cpp}{next} 指针赋值'' 的耗时差异; 另一方面, \mintinline{cpp}{std::list::merge} 的源码经编译器优化后可读性较低, 难以直接追踪执行流程.

后续计划尝试使用 \mintinline{bash}{perf} 工具统计 \mintinline{cpp}{merge} 过程中的内存访问次数与指令周期, 同时手动实现双向链表的 merge 函数, 通过控制变量法 (如暂不维护 \mintinline{cpp}{prev} 指针) 对比耗时, 验证 \mintinline{cpp}{prev} 指针的影响.

\subsubsection{模板类动态类型识别的效率优化}

为统一测试四种容器, 实验中编写了模板函数 \mintinline{cpp}{test<T>(dout)} 与 \mintinline{cpp}{profile<T>(n, dout)}. 当需要根据模板参数 \mintinline{cpp}{T} 动态选择操作 (如对链表跳过 ``随机访问测试'') 时, 采用 \mintinline{cpp}{typeid(T).name()} 进行判断——该操作属于运行时类型识别 (RTTI), 需访问对象的类型信息表, 存在一定开销. 在高频调用场景下, RTTI 操作使整体耗时增加约 \qtyrange[range-phrase={\,\textasciitilde\,}]{3}{5}{\%}.

若采用编译期模板元编程 (如 \mintinline{cpp}{std::enable_if}), 虽可避免 RTTI 开销, 但需为每种容器编写特化版本, 可能降低代码复用性. 后续需进一步探索 ``编译期判断 + 模板特化'' 的平衡方案, 例如通过 \mintinline{cpp}{std::is_same<T, ordered_list<int>>::value} 在编译期区分容器类型, 对比其与 RTTI 方案的代码量、维护成本及性能差异.

\subsection{实验图表的额外应用与认知提升}

本次实验除生成 ``归并耗时随数据规模 $n$ 的变化曲线'' 外, 还绘制了尾部追加元素、有序插入元素、查找元素、删除元素、排序容器、清空容器等操作随数据规模 $n$ 的变化曲线. 这些图表不仅用于验证性能差异, 也帮助加深了对线性表存储特性的理解.

\begin{figure}[H]
    \centering
    \includegraphics[width=\textwidth]{img/profile_push_back.png}
\end{figure}

\begin{figure}[H]
    \centering
    \includegraphics[width=\textwidth]{img/profile_ordered_insert.png}
\end{figure}

\begin{figure}[H]
    \centering
    \includegraphics[width=\textwidth]{img/profile_contains.png}
\end{figure}

\begin{figure}[H]
    \centering
    \includegraphics[width=\textwidth]{img/profile_remove.png}
\end{figure}

\begin{figure}[H]
    \centering
    \includegraphics[width=\textwidth]{img/profile_sort.png}
\end{figure}

\begin{figure}[H]
    \centering
    \includegraphics[width=\textwidth]{img/profile_clear.png}
\end{figure}

\subsubsection{从有序插入曲线理解缓存局部性}

有序插入曲线显示, 顺序表与链表的耗时均随 $n$ 增长呈 $\mathcal{O}(n)$ 趋势 (顺序表因元素移动, 链表因遍历查找), 但前者斜率仅为后者的 $\frac{1}{20}$——$n=100000$ 时, 顺序表有序插入耗时 \qty{0.28}{s}, 链表则为 \qty{39.5}{s}. 这一差异源于 ``连续内存的缓存局部性'':顺序表移动元素时, CPU 会预加载连续内存块到缓存, 批量处理效率高; 而链表遍历需频繁跳转至离散节点, 易触发缓存失效, 导致效率降低. 通过曲线对比, 更直观地认识到存储结构连续性对操作效率的影响, 这比单纯的理论分析更具说服力.

\subsubsection{对比归并与插入曲线分析操作特性}

对比归并曲线与有序插入曲线的斜率发现, 归并操作的常数因子更低: 顺序表归并的斜率仅约为插入斜率的 $\frac{2}{7}$. 原因在于归并仅需一次遍历与复制即可完成有序整合, 而有序插入需对每个元素单独执行 ``查找 + 移动 / 链接'', 多次遍历导致开销累积. 这一发现提示, 设计线性表算法时, 应优先选择 ``批量处理'' 方式, 减少单次操作的遍历次数, 为后续算法优化提供了参考.

综上, 本次实验通过解决实际问题、分析性能图表, 加深了对线性表与 C++ 编程的理解. 待探索的问题也明确了后续学习的方向, 为进一步掌握数据结构与编程优化打下基础.

\end{document}