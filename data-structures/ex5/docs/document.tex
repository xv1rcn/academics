\documentclass[12pt, a4paper]{ctexart}
\usepackage[margin=2cm]{geometry}
\usepackage{graphicx}
\usepackage{libertine}
\usepackage{siunitx}

\usepackage{titlesec}
\usepackage{zhnumber}
\titleformat*{\section}{\Large\bfseries\raggedright}
\renewcommand{\thesection}{\zhnum{section}}
\renewcommand{\thesubsection}{\arabic{subsection}}

\setcounter{secnumdepth}{4}
\renewcommand{\theparagraph}{\thesubsubsection.\arabic{paragraph}}
\renewcommand{\paragraph}[1]{
    \refstepcounter{paragraph}
    {\bfseries\theparagraph\quad#1}\par
    \vspace{2pt}
    \noindent
}

\usepackage{enumitem}
\setlist[enumerate]{itemsep=2pt, parsep=0pt, partopsep=0pt, topsep=2pt}
\setlist[itemize]{itemsep=2pt, parsep=0pt, partopsep=0pt, topsep=2pt}
\linespread{1.2}

\usepackage{minted}
\setminted{
    breaklines=true, escapeinside=||, fontsize=\small, frame=lines,
    mathescape=$$, numbers=none, style=bw, tabsize=4
}

\begin{document}
    
\pagestyle{plain}
\thispagestyle{empty}

\noindent
\begin{tabular*}{\textwidth}{l @{\extracolsep{\fill}} r @{\extracolsep{6pt}} l}
    \LARGE{\textbf{实验报告}} & 数据结构 & \textit{Data Structures} \\
\end{tabular*}\\\\
\begin{tabular*}{\textwidth}{l l}
    \textbf{报告标题: } & \textbf{树算法} \\
    \textbf{学号: } & 19240212 \\
    \textbf{姓名: } & 华博文 \\
    \textbf{日期: } & 2025 年 11 月 11 日 \\
\end{tabular*}\\
\rule[2ex]{\textwidth}{2pt}

\section{实验环境}

\subsection{操作系统}

Ubuntu 24.04.3 LTS x86\_64,Linux 6.14.0-28-generic,AMD Ryzen 7 7700 (16) @ 5.3GHz。

\subsection{编程工具}

NeoVim v0.12.0-dev,gcc 13.3.0。

\subsection{其他工具}

\LaTeX{} (\TeX\textit{studio})。

\section{实验内容及其完成情况}

\subsection{实验目的}

\begin{enumerate}
    \item 掌握树的基本概念和存储结构;
    \item 理解树的遍历方式及其应用;
    \item 掌握树常用操作的算法实现;
    \item 培养使用 C++ 实现复杂数据结构的能力。
\end{enumerate}

\subsection{实验内容}

编写程序,实现树类及若干应用算法。要求采用孩子兄弟表示法,实现以下功能:

\begin{enumerate}
    \item 构造函数:根据序偶集合构造对象;
    \item 析构函数:释放所有结点空间(本实验采用智能指针自动管理);
    \item 先根、后根遍历算法;
    \item 计算每个结点的度;
    \item 计算树的高度;
    \item 输出根结点到每个叶子结点的路径。
\end{enumerate}

\section{实现细节与过程}

本节简要说明实现的关键点与算法契约(输入 / 输出 / 错误处理)。

\subsection{设计契约}

\begin{itemize}
    \item 输入:结点关系的序偶集合(如 \mintinline{c++}`std::vector<std::pair<int, int>>`),每个序偶表示父子关系;
    \item 输出:支持遍历、度、高度、路径等操作的树结构;
    \item 错误模式:输入非法(如环、孤立结点)可通过异常或提示处理。
\end{itemize}

\subsection{数据结构定义}

采用孩子兄弟表示法,核心结点类型如下:

\begin{minted}{c++}
template <typename T> struct TreeNode {
    T data;
    std::unique_ptr<TreeNode<T>> firstChild;
    std::unique_ptr<TreeNode<T>> nextSibling;
    TreeNode(const T &val) : data(val) {}
};
\end{minted}

树类接口如下:

\begin{minted}{c++}
template <typename T> class Tree {
  public:
    Tree(const std::vector<std::pair<T, T>> &pairs);
    void preOrder() const;
    void postOrder() const;
    void printDegrees() const;
    int getHeight() const;
    void printPaths() const;
    const TreeNode<T> *getRoot() const;
};
\end{minted}

\subsection{树的构建与析构}

构造函数根据序偶集合自动建立树结构,采用哈希表辅助查找,智能指针自动释放所有结点空间,无需手动析构。

\subsection{遍历与操作实现}

\begin{itemize}
    \item 先根遍历:先访问根,再递归访问所有孩子;
    \item 后根遍历:递归访问所有孩子,最后访问根;
    \item 结点度:统计每个结点的孩子数;
    \item 树高度:递归计算所有孩子的最大高度;
    \item 路径输出:递归遍历,遇到叶子结点时输出路径。
\end{itemize}

\subsection{测试用例与运行结果}

设计用例如下:

\begin{minted}{c++}
std::vector<std::pair<int, int>> pairs = {
    {1, 2}, {1, 3}, {2, 4}, {2, 5}, {3, 6}, {3, 7}
};
Tree<int> tree(pairs);

std::cout << "先根遍历: ";
tree.preOrder();  // 1 2 4 5 3 6 7 

std::cout << "后根遍历: ";
tree.postOrder(); // 4 5 2 6 7 3 1 

std::cout << "每个结点的度: " << std::endl;
tree.printDegrees();
// 结点 1 的度: 2
// 结点 2 的度: 2
// 结点 4 的度: 0
// 结点 5 的度: 0
// 结点 3 的度: 2
// 结点 6 的度: 0
// 结点 7 的度: 0

std::cout << "树的高度: " << tree.getHeight() << std::endl; // 3

std::cout << "根到每个叶子的路径: " << std::endl;
tree.printPaths();
// 路径: 1 -> 2 -> 4
// 路径: 1 -> 2 -> 5
// 路径: 1 -> 3 -> 6
// 路径: 1 -> 3 -> 7
\end{minted}

实际运行结果与预期一致,所有功能均通过测试。

\section{复杂度分析与讨论}

对主要操作给出时间 / 空间复杂度总结:

\begin{itemize}
    \item 构造树:$\mathcal{O}(n)$,每个结点和边只处理一次。
    \item 遍历(先根 / 后根):$\mathcal{O}(n)$,每个结点访问一次。
    \item 计算度:$\mathcal{O}(n)$,每个结点的孩子只需遍历一次。
    \item 计算高度:$\mathcal{O}(n)$,递归访问所有结点。
    \item 输出路径:$\mathcal{O}(n)$,每条路径只访问一次所有结点。
\end{itemize}

其中 $n$ 为结点总数。空间复杂度最坏也是 $\mathcal{O}(n)$(递归栈或辅助结构)。

如果树是平衡的,递归栈深度为 $\mathcal{O}(\log n)$,退化为链时为 $\mathcal{O}(n)$。

\section{实验总结}

本次实验实现了基于孩子兄弟表示法的树结构,支持多种遍历和常用操作,代码结构清晰,注释完整,测试用例覆盖全面,满足实验要求。采用现代 C++ 编程风格,便于扩展和维护。

\end{document}