\documentclass[12pt, a4paper]{ctexart}
\usepackage[margin=2cm]{geometry}
\usepackage{graphicx}
\usepackage{libertine}
\usepackage{siunitx}

\usepackage{titlesec}
\usepackage{zhnumber}
\titleformat*{\section}{\Large\bfseries\raggedright}
\renewcommand{\thesection}{\zhnum{section}}
\renewcommand{\thesubsection}{\arabic{subsection}}

\setcounter{secnumdepth}{4}
\renewcommand{\theparagraph}{\thesubsubsection.\arabic{paragraph}}
\renewcommand{\paragraph}[1]{
    \refstepcounter{paragraph}
    {\bfseries\theparagraph\quad#1}\par
    \vspace{2pt}
    \noindent
}

\usepackage{enumitem}
\setlist[enumerate]{itemsep=2pt, parsep=0pt, partopsep=0pt, topsep=2pt}
\setlist[itemize]{itemsep=2pt, parsep=0pt, partopsep=0pt, topsep=2pt}
\linespread{1.2}

\usepackage{minted}
\setminted{
    breaklines=true, escapeinside=||, fontsize=\small, frame=lines,
    mathescape=$$, numbers=none, style=bw, tabsize=4
}

\begin{document}
    
\pagestyle{plain}
\thispagestyle{empty}

\noindent
\begin{tabular*}{\textwidth}{l @{\extracolsep{\fill}} r @{\extracolsep{6pt}} l}
    \LARGE{\textbf{实验报告}} & 数据结构 & \textit{Data Structures} \\
\end{tabular*}\\\\
\begin{tabular*}{\textwidth}{l l}
    \textbf{报告标题: } & \textbf{中缀表达式求值} \\
    \textbf{学号: } & 19240212 \\
    \textbf{姓名: } & 华博文 \\
    \textbf{日期: } & 2025 年 10 月 14 日 \\
\end{tabular*}\\
\rule[2ex]{\textwidth}{2pt}

\section{实验环境}

\subsection{操作系统}

Ubuntu 24.04.3 LTS x86\_64,Linux 6.14.0-28-generic,AMD Ryzen 7 7700 (16) @ 5.3GHz。

\subsection{编程工具}

NeoVim v0.11.1,gcc 13.3.0。

\subsection{其他工具}

\LaTeX{} (\TeX\textit{studio})。

\section{实验内容及其完成情况}

\subsection{实验目的}

\begin{enumerate}
    \item \textbf{理解中缀表达式结构与求值规则:} 本实验旨在深入理解中缀表达式的语法结构及其求值规则。中缀表达式是最常见的数学表达式形式,运算符位于两个操作数之间。通过本实验,将掌握运算符优先级、结合性以及括号的处理方法,为后续复杂表达式的计算打下基础。
    \item \textbf{掌握栈在表达式求值中的应用:} 中缀表达式求值通常采用双栈法:一个操作数栈、一个运算符栈。实验过程中,将学习如何利用栈实现表达式的分步计算,巩固栈的基本操作(入栈、出栈、取栈顶等),并体会栈在算法设计中的重要作用。
    \item \textbf{提升复杂逻辑的实现能力:} 实验涉及运算符优先级比较、括号匹配、多位数及小数解析等多种边界情况。通过实现完整的中缀表达式求值算法,将提升处理复杂逻辑和异常情况的能力,增强代码健壮性和容错性。
\end{enumerate}

\subsection{实验内容}

实现一个程序,能够对用户输入的中缀表达式进行求值。采用双栈法(操作数栈与运算符栈),支持基本四则运算和括号,保证程序健壮性,能正确处理非法输入。功能要求如下:

\begin{enumerate}
    \item 支持运算符:\mintinline{c++}`+`、\mintinline{c++}`-`、\mintinline{c++}`*`、\mintinline{c++}`/`;
    \item 支持括号:\mintinline{c++}`(`、\mintinline{c++}`)`;
    \item 支持多位数及小数(如 \mintinline{c++}`42.57`);
    \item 对非法表达式(括号不匹配、运算符错误、除零等)能给出明确提示;
    \item 采用 C++ 语言实现。
\end{enumerate}

\subsection{实验过程}

\subsubsection{表达式分词 \mintinline{c++}`tokenize_expression`}

表达式分词是整个求值流程的第一步。\mintinline{c++}`tokenize_expression` 函数负责将用户输入的字符串拆分为有类型的 \mintinline{c++}`Token`,包括数字、运算符和括号。分词时,程序会逐字符扫描输入,遇到数字或小数点时,会连续读取形成完整的多位数或小数,并通过 \mintinline{c++}`std::stod` 转换为 \mintinline{c++}`double` 类型。例如,输入 \mintinline{c++}`42.57 + 3` 会被分解为 \mintinline{c++}`Token(NUMBER, 42.57)`、\mintinline{c++}`Token(OPERATOR, '+')` 和 \mintinline{c++}`Token(NUMBER, 3)`。

分词过程中还会检测数字格式错误(如多重小数点),并对非法字符进行处理。如果发现数字中有多重小数点,或出现非法字符,程序会抛出异常并提示错误。例如:

\begin{minted}{c++}
if (expr[i] == '.') {
    if (dot) throw std::runtime_error(ERR_MULTIPLE_DOT);
    dot = true;
}
...
else {
    throw std::runtime_error(ERR_ILLEGAL_CHAR + expr[i]);
}
\end{minted}

此外,括号和运算符也会被正确识别为独立的 \mintinline{c++}`Token`,便于后续处理:

\begin{minted}{c++}
if (std::string("+-*/").find(expr[i]) != std::string::npos) {
    token_list.push_back(Token(OPERATOR, expr[i]));
    ++i;
} else if (expr[i] == '(' || expr[i] == ')') {
    token_list.push_back(Token(PARENTHESIS, expr[i]));
    ++i;
}
\end{minted}

\subsubsection{运算符优先级与运算 \mintinline{c++}`get_precedence`、\mintinline{c++}`apply_operator`}

运算符优先级决定了计算的顺序。\mintinline{c++}`get_precedence` 函数用于返回运算符的优先级,乘除高于加减:

\begin{minted}{c++}
int get_precedence(char op) {
    if (op == '+' || op == '-') return 1;
    if (op == '*' || op == '/') return 2;
    return 0;
}
\end{minted}

\mintinline{c++}`apply_operator` 函数负责实际的四则运算操作。它根据传入的运算符类型,对两个操作数进行加、减、乘、除运算。特别地,除法运算时会检测除数是否为零,若为零则抛出 \mintinline{c++}`ERR_DIV_ZERO` 异常,保证程序健壮性:

\begin{minted}{c++}
double apply_operator(double a, double b, char op) {
    switch (op) {
        case '+':
            return a + b;
        case '-':
            return a - b;
        case '*':
            return a * b;
        case '/':
            if (b == 0) throw std::runtime_error(ERR_DIV_ZERO);
            return a / b;
        default:
            throw std::runtime_error(ERR_UNKNOWN_OP);
    }
}
\end{minted}

\subsubsection{表达式求值 \mintinline{c++}`evaluate_expression`}

\mintinline{c++}`evaluate_expression` 是核心计算函数,实现了双栈法。遍历分词得到的 \mintinline{c++}`Token` 列表时,遇到数字直接压入操作数栈 \mintinline{c++}`value_stack`,遇到运算符则根据优先级与栈顶运算符比较,若优先级较低或相等,则弹出栈顶运算符并进行计算,将结果压入操作数栈。例如,处理 \mintinline{c++}`3 + 4 * 2` 时,乘法优先级高于加法,先计算 \mintinline{c++}`4 * 2`,再与 \mintinline{c++}`3` 相加。

括号的处理则更为复杂,遇到左括号直接压入运算符栈,遇到右括号则不断弹出运算符并计算,直到遇到匹配的左括号为止。如果括号不匹配,程序会抛出 \mintinline{c++}`ERR_PAREN_MISMATCH` 错误。如下代码片段展示了括号处理逻辑:

\begin{minted}{c++}
if (token.type == PARENTHESIS) {
    if (token.op == '(') {
        op_stack.push('(');
    } else if (token.op == ')') {
        bool found_left = false;
        while (!op_stack.empty()) {
            if (op_stack.top() == '(') {
                op_stack.pop();
                found_left = true;
                break;
            } else {
                if (value_stack.size() < 2) throw std::runtime_error(ERR_MISSING_OPERAND);
                double b = value_stack.top();
                value_stack.pop();
                double a = value_stack.top();
                value_stack.pop();
                char op = op_stack.top();
                op_stack.pop();
                value_stack.push(apply_operator(a, b, op));
            }
        }
        if (!found_left) throw std::runtime_error(ERR_PAREN_MISMATCH);
    }
}
\end{minted}

表达式处理完毕后,程序会依次弹出剩余运算符并计算,最终操作数栈应只剩一个结果。若栈中多余操作数或操作数不足,分别抛出 \mintinline{c++}`ERR_EXTRA_OPERAND` 或 \mintinline{c++}`ERR_MISSING_OPERAND` 错误,确保表达式结构正确:

\begin{minted}{c++}
while (!op_stack.empty()) {
    if (op_stack.top() == '(' || op_stack.top() == ')') throw std::runtime_error(ERR_PAREN_MISMATCH);
    if (value_stack.size() < 2) throw std::runtime_error(ERR_MISSING_OPERAND);
    double b = value_stack.top();
    value_stack.pop();
    double a = value_stack.top();
    value_stack.pop();
    char op = op_stack.top();
    op_stack.pop();
    value_stack.push(apply_operator(a, b, op));
}
if (value_stack.size() != 1) throw std::runtime_error(ERR_EXTRA_OPERAND);
\end{minted}

\subsubsection{主程序与异常处理}

主函数负责与用户交互,循环读取输入表达式,调用分词和求值函数,并输出结果或错误提示。所有异常均在主函数中捕获,并统一输出为 \mintinline{text}`[错误] ...` 格式,提升用户体验。如下代码片段展示了主循环的异常处理:

\begin{minted}{c++}
try {
    auto token_list = tokenize_expression(input_line);
    double result = evaluate_expression(token_list);
    std::cout << " = " << result << std::endl;
} catch (const std::exception &e) {
    std::cout << "[错误] " << e.what() << std::endl;
}
\end{minted}

用户可多次输入表达式,按 Ctrl+C 退出程序。整个流程实现了对中缀表达式的高效、健壮求值,支持多位数、小数、括号嵌套及异常处理,充分体现了栈结构在表达式计算中的应用价值。

\subsection{实验结果与分析}

本实验实现的中缀表达式求值程序在功能和健壮性方面表现良好。经过多组测试,程序能够正确处理各种合法表达式,包括带括号的嵌套运算、多位数和小数。例如,输入 \mintinline{text}`3 + 4 * (2 - 1)`,程序会先分词为数字和运算符,然后根据优先级和括号规则,先计算括号内的 \mintinline{text}`2 - 1`,再进行乘法和加法,最终输出 \mintinline{text}`= 7`。对于小数运算如 \mintinline{text}`42.57 / 2`,分词和计算均能准确完成,输出 \mintinline{text}`= 21.285`。

在异常处理方面,程序对非法输入有明确提示。例如,输入 \mintinline{text}`3 + (4 * 2` 时,由于括号不匹配,程序会抛出异常并输出 \mintinline{text}`[错误] 括号不匹配`。输入 \mintinline{text}`5 / 0` 时,检测到除零操作,输出 \mintinline{text}`[错误] 除零错误`。对于包含非法字符的表达式如 \mintinline{text}`2 + a`,分词阶段会捕获并提示 \mintinline{text}`[错误] 非法字符: a`。这些错误处理均通过如下代码实现:

\begin{minted}{c++}
catch (const std::exception &e) {
    std::cout << "[错误] " << e.what() << std::endl;
}
\end{minted}

此外,程序对表达式结构的异常也有检测。例如,输入 \mintinline{text}`1 +` 或 \mintinline{text}`1 2 +`,会分别提示“表达式缺少操作数”或“表达式多余操作数”,保证结果的唯一性和正确性。整体来看,程序不仅能正确计算结果,还能对各种边界情况和错误输入做出合理响应,体现了良好的健壮性和用户体验。

\section{实验总结}

通过本次实验,对中缀表达式的结构和求值规则有了深入理解,尤其是运算符优先级、结合性以及括号的处理。实验过程中,双栈法的应用使表达式求值过程变得清晰和高效,进一步巩固了对栈这一数据结构的掌握。分词、优先级判断、括号匹配和异常处理等环节的实现,锻炼了复杂逻辑的编程能力。

代码实现过程中,采用了现代 C++ 风格,结构清晰,命名规范,错误信息集中管理,便于维护和扩展。异常处理机制保证了程序的健壮性,即使面对各种非法输入也能给出明确提示,避免程序崩溃。主循环设计使用户可以多次输入表达式,交互体验良好。

总的来说,本实验不仅完成了中缀表达式求值的功能目标,还提升了对数据结构和算法设计的理解与应用能力。通过实际编程和测试,体会到细致的边界处理和错误检测对于高质量程序的重要性,为后续学习更复杂的表达式处理和数据结构应用打下了坚实基础。

\end{document}