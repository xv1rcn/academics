\documentclass[12pt, a4paper]{ctexart}
\usepackage[margin=2cm]{geometry}
\usepackage{graphicx}
\usepackage{libertine}
\usepackage{siunitx}

\usepackage{titlesec}
\usepackage{zhnumber}
\titleformat*{\section}{\Large\bfseries\raggedright}
\renewcommand{\thesection}{\zhnum{section}}
\renewcommand{\thesubsection}{\arabic{subsection}}

\setcounter{secnumdepth}{4}
\renewcommand{\theparagraph}{\thesubsubsection.\arabic{paragraph}}
\renewcommand{\paragraph}[1]{
    \refstepcounter{paragraph}
    {\bfseries\theparagraph\quad#1}\par
    \vspace{2pt}
    \noindent
}

\usepackage{enumitem}
\setlist[enumerate]{itemsep=2pt, parsep=0pt, partopsep=0pt, topsep=2pt}
\setlist[itemize]{itemsep=2pt, parsep=0pt, partopsep=0pt, topsep=2pt}
\linespread{1.2}

\usepackage{minted}
\setminted{
    breaklines=true, escapeinside=||, fontsize=\small, frame=lines,
    mathescape=$$, numbers=none, style=bw, tabsize=4
}

\begin{document}
    
\pagestyle{plain}
\thispagestyle{empty}

\noindent
\begin{tabular*}{\textwidth}{l @{\extracolsep{\fill}} r @{\extracolsep{6pt}} l}
    \LARGE{\textbf{实验报告}} & 数据结构 & \textit{Data Structures} \\
\end{tabular*}\\\\
\begin{tabular*}{\textwidth}{l l}
    \textbf{报告标题: } & \textbf{二叉树算法} \\
    \textbf{学号: } & 19240212 \\
    \textbf{姓名: } & 华博文 \\
    \textbf{日期: } & 2025 年 11 月 1 日 \\
\end{tabular*}\\
\rule[2ex]{\textwidth}{2pt}

\section{实验环境}

\subsection{操作系统}

Ubuntu 24.04.3 LTS x86\_64,Linux 6.14.0-28-generic,AMD Ryzen 7 7700 (16) @ 5.3GHz。

\subsection{编程工具}

NeoVim v0.11.1,gcc 13.3.0,GNU Make 4.3,cmake 3.28.3。

\subsection{其他工具}

\LaTeX{} (\TeX\textit{studio})。

\section{实验内容及其完成情况}

\subsection{实验目的}

\begin{enumerate}
    \item 掌握二叉树的基本概念和存储结构;
    \item 理解二叉树的四种遍历方式及其应用:先序、中序、后序、层次遍历;
    \item 掌握二叉树常用操作(创建、遍历、求高度、计数、查找、求父结点)的算法实现;
    \item 提高使用 C++ 面向对象实现复杂数据结构的能力。
\end{enumerate}

\subsection{实验内容}

实现一个完整的二叉树类,包含以下功能:

\begin{enumerate}
    \item 根据带虚结点(如使用特殊符号表示空子结点)的先序序列创建二叉树;
    \item 四种遍历:先序、(递归 / 非递归)中序、后序和层次遍历;
    \item 求二叉树的高度和结点总数;
    \item 查找指定结点并返回该结点的父结点指针或值(若存在)。
\end{enumerate}

\section{实现细节与过程}

本节简要说明实现的关键点与算法契约(输入 / 输出 / 错误处理)。

\subsection{设计契约}

\begin{itemize}
    \item 输入:使用带虚结点的先序序列(例如:\mintinline{text}`AB##CD##E##`),其中 \mintinline{text}`#` 表示空结点;
    \item 输出:构建出的二叉树,支持各种遍历输出、查询高度、结点数、查找结点及其父结点;
    \item 错误模式:非法输入(例如序列过短或不匹配)将通过异常或返回特殊值提示调用者。
\end{itemize}

\subsection{数据结构定义}

核心数据类型为二叉树结点(C++ 结构体或类内部私有类型):

\begin{minted}{c++}
template <typename T>
class BinaryTree {
public:
    struct Node {
        T val;
        std::unique_ptr<Node> left;
        std::unique_ptr<Node> right;
        explicit Node(const T &v) : val(v), left(nullptr), right(nullptr) {}
    };

private:
    ...
};
\end{minted}

外层提供 \mintinline{c++}`BinaryTree` 类,封装创建与各类操作,并在析构时释放内存。

\subsection{从带虚结点的先序序列创建二叉树}

算法思路:使用迭代器或索引从序列头部递归构建,遇到虚结点符号(例如 \mintinline{text}`#`)返回空。

递归伪代码:

\begin{minted}{c++}
std::unique_ptr<Node> buildFromPreorderRec(const std::vector<T> &tokens, size_t &idx, const T &nullValue) {
    if (idx >= tokens.size())
        return nullptr;
    T cur = tokens[idx++];
    if (cur == nullValue)
        return nullptr;
    auto node = std::make_unique<Node>(cur);
    node->left = buildFromPreorderRec(tokens, idx, nullValue);
    node->right = buildFromPreorderRec(tokens, idx, nullValue);
    return node;
}
\end{minted}

时间复杂度:$\mathcal{O}\left(n\right)$,空间复杂度:递归栈最坏 $\mathcal{O}\left(n\right)$。

\subsection{遍历实现}

先序 / 中序 / 后序遍历均提供递归实现,并在需要时提供基于栈的非递归版本(中序与先序常用)。层次遍历使用队列实现(BFS)。

\begin{minted}{c++}
std::vector<T> levelOrderRec(Node *node) const {
    std::vector<T> out;
    if (!node)
        return out;
    std::queue<Node *> q;
    q.push(node);
    while (!q.empty()) {
        Node *cur = q.front();
        q.pop();
        out.push_back(cur->val);
        if (cur->left)
            q.push(cur->left.get());
        if (cur->right)
            q.push(cur->right.get());
    }
    return out;
}
\end{minted}

时间复杂度:每种遍历都访问每个结点一次,均为 $\mathcal{O}\left(n\right)$,空间复杂度:层次遍历 $\mathcal{O}\left(\texttt{width}\right)$(最坏为 $\mathcal{O}\left(n\right)$),递归版本的额外空间为递归深度 $\mathcal{O}\left(h\right)$。

\subsection{高度与结点计数}

使用递归方式计算高度与结点总数:

\begin{minted}{c++}
int heightRec(Node *node) const {
    if (!node)
        return 0;
    return 1 + std::max(heightRec(node->left.get()), heightRec(node->right.get()));
}

int countRec(Node *node) const {
    if (!node)
        return 0;
    return 1 + countRec(node->left.get()) + countRec(node->right.get());
}
\end{minted}

时间复杂度均为 $\mathcal{O}\left(n\right)$,空间复杂度最坏 $\mathcal{O}\left(n\right)$(退化链)。

\subsection{查找指定结点与求父结点}

查找结点:可使用任一遍历(递归或非递归)在 $\mathcal{O}\left(n\right)$ 时间找到第一个匹配值的结点指针。

查找父结点:在遍历过程中记录父指针,或在 \mintinline{c++}`BinaryTree` 中维护父指针(在创建时设定)。下面给出递归查找父结点的示例:

\begin{minted}{c++}
Node *findParentRec(Node *node, Node *parent, const T &value) const {
    if (!node)
        return nullptr;
    if (node->val == value)
        return parent;
    Node *l = findParentRec(node->left.get(), node, value);
    if (l)
        return l;
    return findParentRec(node->right.get(), node, value);
}
\end{minted}

时间复杂度:$\mathcal{O}\left(n\right)$。

\subsection{测试用例与运行结果}

设计了若干用例覆盖常见场景:空树、单结点、完全二叉树、非完全二叉树以及包含虚结点的序列。示例输入与期望输出:

\begin{enumerate}
    \item 输入先序序列:\mintinline{text}`ABD##E##C##`(对应的树为:A 的左子树为 B,B 的左子树为 D,右子树为 E;A 的右子树为 C)
    \item 先序输出:A B D E C
    \item 中序输出:D B E A C
    \item 后序输出:D E B C A
    \item 层次输出:A B C D E
    \item 结点总数:5,高度:3
    \item 查找结点 E 返回存在;查找父结点返回 B。
\end{enumerate}

在我的本地测试中,上述用例均输出与预期一致,边界用例(全为 \mintinline{text}`#` 的序列表示空树)亦能正确处理。

\section{复杂度分析与讨论}

对主要操作给出时间/空间复杂度总结:

\begin{itemize}
    \item 构建二叉树(先序带虚结点):时间 $\mathcal{O}\left(n\right)$,空间 $\mathcal{O}\left(n\right)$(用于递归栈或临时序列);
    \item 各种遍历(先/中/后/层次):时间均为 $\mathcal{O}\left(n\right)$,空间最坏 $\mathcal{O}\left(n\right)$(递归深度或队列/栈);
    \item 计算高度/计数:时间 $\mathcal{O}\left(n\right)$,空间 $\mathcal{O}\left(h\right)$(递归栈);
    \item 查找结点 / 查找父结点:时间 $\mathcal{O}\left(n\right)$,空间 $\mathcal{O}\left(h\right)$(递归)或 $\mathcal{O}\left(1\right)$(迭代并记录)。
\end{itemize}

常见优化与注意点:

\begin{itemize}
    \item 若二叉树是平衡的,递归栈深度为 $\mathcal{O}\left(\log{n}\right)$;退化链情况退化到 $\mathcal{O}\left(n\right)$。
    \item 若频繁查找父结点,可在结点结构中维护父指针,查找父结点复杂度可降为 $\mathcal{O}\left(1\right)$(若已有结点指针);
    \item 若结点值不是字符而是复杂对象,建议使用模板类并传入比较函数或重载比较操作符以支持泛型。
\end{itemize}

\section{实验总结}

本次实验实现了一个功能完整的二叉树类,并验证了:

\begin{itemize}
    \item 能根据带虚结点的先序序列正确构建二叉树;
    \item 提供递归与非递归的遍历实现,层次遍历使用队列实现;
    \item 能正确计算结点总数和树的高度;
    \item 能查找指定结点并返回其父结点(递归实现),并对复杂度给出分析。
\end{itemize}

代码结构清晰,注释完整;测试用例覆盖了常见与边界场景。该实现为进一步扩展(例如添加删除结点、子树复制、序列化/反序列化等)提供了良好基础。本报告按要求完成了二叉树的设计、实现与测试,并对复杂度进行了分析,满足实验要求。

\end{document}